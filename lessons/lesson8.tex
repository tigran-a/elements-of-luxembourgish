\chapter{Leçon 8}

\section{Gespréich a Froen / Conversation and Questions / Conversation et Questions}

\subsection{Den Dag beschreiwen / Describing the Day / Décrire la Journée}

\begin{itemize}
    \item \textbf{Wéi war däin Dag?} \\
    \textit{(Comment s'est passée ta journée ? / How was your day?)}
    
    \item \textbf{Gutt. / Ganz gutt.} \\
    \textit{(Bien. / Très bien. / Good. / Very good.)}
    
    \item \textbf{Net esou gutt.} \\
    \textit{(Pas très bien. / Not so good.)}
    
    \item \textbf{Esou lala.} \\
    \textit{(Comme ci, comme ça. / So-so.)}
    
    \item \textbf{Ça va.} \\
    \textit{(Ça va. / It's okay.)}
    
    \item \textbf{Ech si midd, well ech haut vill geschafft hunn.} \\
    \textit{(Je suis fatigué(e) parce que j'ai beaucoup travaillé aujourd'hui. / I am tired because I worked a lot today.)}
    
    \item \textbf{Gëschter hunn ech vun 8 Auer moies bis 11 Auer owes geschafft.} \\
    \textit{(Hier, j'ai travaillé de 8 heures du matin à 11 heures du soir. / Yesterday I worked from 8 am till 11 pm.)}
    
    \item \textbf{Wéi vill Stonnen schaffs du?} \\
    \textit{(Combien d'heures travailles-tu ? / How many hours do you work?)}
    
    \item \textbf{Ech schaffen 8 Stonnen den Dag.} \\
    \textit{(Je travaille 8 heures par jour. / I work 8 hours a day.)}
\end{itemize}

\subsection{Hierkonft a Reesen / Origins and Travel / Origines et Voyages}

\begin{itemize}
    \item \textbf{Wou kommt Dir hier? (Formal) / Wou kënns du hier? (Informal)} \\
    \textbf{Vu wou kommt Dir? (Formal) / Vu wou kënns du? (Informal)} \\
    \textit{(D'où venez-vous ? / D'où viens-tu ? / Where are you from?)}
    
    \item \textbf{Ech kommen aus Syrien, vun Damaskus.} \\
    \textit{(Je viens de Syrie, de Damas. / I am from Syria, from Damascus.)}
    
    \item \textbf{Ech kommen aus Portugal.} \\
    \textit{(Je viens du Portugal. / I am from Portugal.)}
    
    \item \textbf{Aus wéi enger Stad? / Aus wat fir enger Stad?} \\
    \textit{(De quelle ville ? / From which city?)}
    
    \item \textbf{Wéi sidd Dir heihi komm? / Wéi bass du heihi komm?} \\
    \textit{(Comment êtes-vous venu ici ? / Comment es-tu venu ici ? / How did you come here?)}
    
    \item \textbf{Ech si mam Auto komm.} \\
    \textit{(Je suis venu en voiture. / I came by car.)}
    
    \item \textbf{Ech sinn zu Fouss komm.} \\
    \textit{(Je suis venu à pied. / I came by foot.)}
    
    \item \textbf{Ech si mam Vëlo komm.} \\
    \textit{(Je suis venu à vélo. / I came by bike.)}
\end{itemize}

\subsection{Grënn an Datumer / Reasons and Dates / Raisons et Dates}

\begin{itemize}
    \item \textbf{Firwat sinn ech hei?} \\
    \textit{(Pourquoi suis-je ici ? / Why am I here?)}
    
    \item \textbf{Ech mengen, du wëlls Lëtzebuergesch léieren.} \\
    \textit{(Je pense que tu veux apprendre le luxembourgeois. / I think you want to learn Luxembourgish.)}
    
    \item \textbf{Wéini ass däi Gebuertsdag?} \\
    \textit{(C'est quand ton anniversaire ? / When is your birthday?)}
    
    \item \textbf{De 24. Oktober 1996 (De véieranzwanzegsten Oktober...)} \\
    \textit{(Le 24 octobre 1996. / The 24th of October 1996.)}
    
    \item \textbf{Mäi Gebuertsdag war virgëschter.} \\
    \textit{(Mon anniversaire était avant-hier. / My birthday was the day before yesterday.)}
\end{itemize}

\subsection{Iessen / Food / Nourriture}

\begin{itemize}
    \item \textbf{Wat ëss du gär?} \\
    \textit{(Qu'aimes-tu manger ? / What do you like to eat?)}
    
    \item \textbf{Ech iesse gär Nuddelen a Pizza.} \\
    \textit{(J'aime manger des pâtes et de la pizza. / I like eating pasta and pizza.)}
    
    \item \textbf{Ech iesse gär Geméis an Uebst.} \\
    \textit{(J'aime manger des légumes et des fruits. / I like eating vegetables and fruits.)}
\end{itemize}

\section{Grammaire / Grammar / Grammaire}

\subsection{Hei/Heihin vs. Do/Dohin}
In Luxembourgish, there is a distinction between \textbf{location} (where you are) and \textbf{direction} (where you are going).
En luxembourgeois, il existe une distinction entre la \textbf{localisation} (où vous êtes) et la \textbf{direction} (où vous allez).

\begin{tabular}{l l l}
\toprule
\textbf{Lëtzebuergesch} & \textbf{Français} & \textbf{English} \\
\midrule
\textbf{Hei} & Ici (Localisation) & Here (Location) \\
\textbf{Heihin / Heihinner} & Vers ici (Direction) & To here (Direction) \\
\textbf{Do} & Là-bas (Localisation) & There (Location) \\
\textbf{Dohin / Dohinner} & Vers là-bas (Direction) & To there (Direction) \\
\bottomrule
\end{tabular}

\vspace{0.5em}
\textbf{Beispiller / Exemples:}
\begin{itemize}
    \item Ech sinn \textbf{hei}. (Je suis ici. / I am here.)
    \item Komm \textbf{heihin}! (Viens ici ! / Come here!)
    \item Den Auto steet \textbf{do}. (La voiture est là-bas. / The car is there.)
    \item Ech ginn \textbf{dohin}. (Je vais là-bas. / I am going there.)
\end{itemize}

\subsection{Possessivpronomen / Possessive Pronouns / Pronoms Possessifs}
The ending of the possessive pronoun depends on the gender and number of the noun that follows it.
La terminaison du pronom possessif dépend du genre et du nombre du nom qui le suit.

\begin{tabular}{l l l l}
\toprule
\textbf{Besëtzer (Propriétaire / Owner)} & \textbf{Masc. / Neut.} & \textbf{Fem. / Pl.} & \textbf{Trad./Trans. (M/F)} \\
\midrule
Ech (I) & \textbf{mäin} & \textbf{meng} & mon/ma (my) \\
Du (You) & \textbf{däin} & \textbf{deng} & ton/ta (your) \\
Hien (He) & \textbf{säin} & \textbf{seng} & son/sa (his) \\
Hatt (She - girl) & \textbf{säin} & \textbf{seng} & son/sa (her) \\
Si (She - woman) & \textbf{hir} & \textbf{hir} & son/sa (her) \\
Mir (We) & \textbf{eisen} & \textbf{eis} & notre/nos (our) \\
Dir (You plural/formal) & \textbf{ären}/\textbf{Ären} & \textbf{är}/\textbf{Är} & votre/vos (your) \\
Si (They) & \textbf{hiren} & \textbf{hir} & leur/leurs (their) \\
\bottomrule
\end{tabular}

\vspace{0.5em}
\textbf{Beispiller:}
\begin{itemize}
    \item \textbf{Mäin} Numm (M) / \textbf{Meng} Schwëster (F).
    \item \textbf{Däin} Auto (M) / \textbf{Deng} Valise (F).
    \item \textbf{Ären} Dag (M) / \textbf{Är} Famill (F).
\end{itemize}

\subsection{Zwee vs. Zwou / Deux (Masculin/Neutre vs. Féminin)}
The number "two" changes depending on the gender of the noun.
Le nombre "deux" change en fonction du genre du nom.

\begin{itemize}
    \item \textbf{Zwee}: Used for Masculine and Neuter nouns. / Utilisé pour les noms masculins et neutres.
    \item \textbf{Zwou}: Used for Feminine nouns. / Utilisé pour les noms féminins.
\end{itemize}

\textbf{Beispiller / Exemples:}
\begin{itemize}
    \item \textbf{Zwee} Männer (Two men - Masc.)
    \item \textbf{Zwee} Kanner (Two children - Neut.)
    \item \textbf{Zwou} Fraen (Two women - Fem.)
    \item \textbf{Zwou} Kazen (Two cats - Fem.)
    \item \textbf{Zwee} Deeg (Two days - Masc.)
    \item \textbf{Zwou} Wochen (Two weeks - Fem.)
\end{itemize}

\section{Dictionary: Iessen / Food / Alimentation}
\begin{multicols}{2}

\begin{minipage}[t]{0.48\textwidth}
    \centering
    \textbf{Nuddelen}
    \vspace{0.5em}
    \framebox{\parbox{0.9\linewidth}{\centering
            \vspace{0.5cm}
            \includegraphics[width=0.9\linewidth]{vokab_300dpi/nuddelen.png} \\
            \small\textit{(pâtes / pasta)}
            \vspace{0.5cm}
    }}
\end{minipage}

\begin{minipage}[t]{0.48\textwidth}
    \centering
    \textbf{eng Pizza}
    \vspace{0.5em}
    \framebox{\parbox{0.9\linewidth}{\centering
            \vspace{0.5cm}
            \includegraphics[width=0.9\linewidth]{vokab_300dpi/pizza.png} \\
            \small\textit{(une pizza / a pizza)}
            \vspace{0.5cm}
    }}
\end{minipage}

\begin{minipage}[t]{0.48\textwidth}
    \centering
    \textbf{Geméis}
    \vspace{0.5em}
    \framebox{\parbox{0.9\linewidth}{\centering
            \vspace{0.5cm}
            \includegraphics[width=0.9\linewidth]{vokab_300dpi/gemies.png} \\
            \small\textit{(légumes / vegetables)}
            \vspace{0.5cm}
    }}
\end{minipage}

\begin{minipage}[t]{0.48\textwidth}
    \centering
    \textbf{Uebst}
    \vspace{0.5em}
    \framebox{\parbox{0.9\linewidth}{\centering
            \vspace{0.5cm}
            \includegraphics[width=0.9\linewidth]{vokab_300dpi/uebst.png} \\
            \small\textit{(fruits / fruit)}
            \vspace{0.5cm}
    }}
\end{minipage}

\begin{minipage}[t]{0.48\textwidth}
    \centering
    \textbf{Brout}
    \vspace{0.5em}
    \framebox{\parbox{0.9\linewidth}{\centering
            \vspace{0.5cm}
            \includegraphics[width=0.9\linewidth]{vokab_300dpi/brout.png} \\
            \small\textit{(pain / bread)}
            \vspace{0.5cm}
    }}
\end{minipage}

\begin{minipage}[t]{0.48\textwidth}
    \centering
    \textbf{Kéis}
    \vspace{0.5em}
    \framebox{\parbox{0.9\linewidth}{\centering
            \vspace{0.5cm}
            \includegraphics[width=0.9\linewidth]{vokab_300dpi/keis.png} \\
            \small\textit{(fromage / cheese)}
            \vspace{0.5cm}
    }}
\end{minipage}

\end{multicols}