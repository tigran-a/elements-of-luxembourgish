\chapter{Leçon 1}

	\section{Le Verbe \textbf{Hunn} (Avoir) - Conjugaison}
	Le verbe \textbf{hunn} est fondamental et irrégulier. Notez que la forme du passé simple (\textbf{Präteritum}) est couramment utilisée en Luxembourgeois.
	
	\subsection*{A. Présent (Präsens)}
	\begin{tabular}{ll c l l}
		\toprule
		\textbf{Pronom} & \textbf{Lëtzebuergesch} & \textbf{} & \textbf{Français} & \textbf{English} \\
		\midrule
		ech & hu(nn) & $\rightarrow$ & J'ai & I have \\
		du & hues & $\rightarrow$ & Tu as & You have \\
		hie(n) / hatt / et & huet & $\rightarrow$ & Il / Elle / On a & He / She / It has \\
		\midrule
		mir & hu(nn) & $\rightarrow$ & Nous avons & We have \\
		dir & hutt & $\rightarrow$ & Vous avez & You have \\
		si & hu(nn) & $\rightarrow$ & Ils / Elles ont & They have \\
		\bottomrule
	\end{tabular}
	
	\subsection*{B. Imparfait (Präteritum)}
	\begin{tabular}{ll c l l}
		\toprule
		\textbf{Pronom} & \textbf{Lëtzebuergesch} & \textbf{} & \textbf{Français (Avais)} & \textbf{English (Had)} \\
		\midrule
		ech & hat & $\rightarrow$ & J'avais & I had \\
		du & has & $\rightarrow$ & Tu avais & You had \\
		hien / hatt / et & hat & $\rightarrow$ & Il / Elle / On avait & He / She / It had \\
		\midrule
		mir & hate(n) & $\rightarrow$ & Nous avions & We had \\
		dir & hat & $\rightarrow$ & Vous aviez & You had \\
		si & hate(n) & $\rightarrow$ & Ils / Elles avaient & They had \\
		\bottomrule
	\end{tabular}
	
	\section{Le Verbe \textbf{Sinn} (Être) - Conjugaison}
	\subsection*{A. Présent (Präsens)}
	\begin{tabular}{ll c l l}
		\toprule
		\textbf{Pronom} & \textbf{Lëtzebuergesch} & \textbf{} & \textbf{Français} & \textbf{English} \\
		\midrule
		ech & si(nn) & $\rightarrow$ & Je suis & I am \\
		du & bass & $\rightarrow$ & Tu es & You are \\
		hien / hatt / et & ass & $\rightarrow$ & Il / Elle / On est & He / She / It is \\
		\midrule
		mir & si(nn) & $\rightarrow$ & Nous sommes & We are \\
		dir & sidd & $\rightarrow$ & Vous êtes & You are \\
		si & si(nn) & $\rightarrow$ & Ils / Elles sont & They are \\
		\bottomrule
	\end{tabular}
	
	\section{Phrases Clés et Adjectifs}
	
	\subsection*{A. Utilisation \textbf{Hunn} (Avoir/to have)}
	\begin{itemize}
		\item \textbf{Ech hunn dech gär.} \quad $\rightarrow$ Je t'aime bien / Je t'apprécie. (I like you.)
		\item \textbf{Mir hunn Honger.} \quad $\rightarrow$ Nous avons faim. (We are hungry, lit. "We have hunger").
		\item \textbf{Du hues néng Joer.} \quad $\rightarrow$ Tu as neuf ans. (You are nine years old.)
		\item \textbf{Hatt huet Angscht.} \quad $\rightarrow$ Elle a peur. (She is afraid, lit. "She has fear").
		\item \textbf{Hien huet de Schnapp.} \quad $\rightarrow$ Il a un rhume. (He has a cold/sniffles).
	\end{itemize}
	
	\subsection*{B. Adjectifs de Base avec \textbf{Sinn} (Être/to be)}
	\begin{itemize}
		\item \textbf{Ech si midd.} \quad $\rightarrow$ Je suis fatigué(e). (I am tired.)
		\item \textbf{Du bass krank.} \quad $\rightarrow$ Tu es malade. (You are sick.)
		\item \textbf{Mir si frou.} \quad $\rightarrow$ Nous sommes contents. (We are happy.)
		\item \textbf{Si sinn traureg.} \quad $\rightarrow$ Ils sont tristes. (They are sad.)
	\end{itemize}
	
	\vspace{1em}\color{geminiBlue}\hrule height 1.5pt \vspace{0.5em}\color{black}
	
	\section{Vocabulaire Illustré : Articles et Pluralité}

	
	\subsection*{A. Le Café (\textbf{Kaffi})}
	\begin{tabular}{l l l l}
		\toprule
		\textbf{Forme} & \textbf{Lëtzebuergesch} & \textbf{Français} & \textbf{English} \\
		\midrule
		Singulier Indéfini & \textbf{e Kaffi} & Un café & A coffee \\
		Singulier Défini & \textbf{de Kaffi} & Le café & The coffee \\
		Pluriel Défini & \textbf{d'Kaffien} & Les cafés & The coffees \\
		\bottomrule
	\end{tabular}
	
	\subsection*{B. La Pâtisserie (\textbf{D'Mëtsch})}
	\begin{tabular}{l l l l}
		\toprule
		\textbf{Forme} & \textbf{Lëtzebuergesch} & \textbf{Français} & \textbf{English} \\
		\midrule
		Singulier Indéfini & \textbf{eng Mëtsch} & Une viennoiserie & A pastry \\
		Singulier Défini & \textbf{d'Mëtsch} & La viennoiserie & The pastry \\
		Pluriel Défini & \textbf{d'Mëtschen} & Les viennoiseries & The pastries \\
		\bottomrule
	\end{tabular}
	
	\begin{minipage}[t]{0.48\textwidth}
		\centering
		\textbf{Singulier}
		\vspace{0.5em}
		\framebox{\parbox{0.9\linewidth}{\centering
				\vspace{0.5cm}
				\includegraphics[width=0.9\linewidth,clip, trim=0 5cm 0 5cm ]{vokab_300dpi/metsch.png} \\
				\small\textit{E Mëtsch (Une Viennoiserie)}
				\vspace{0.5cm}
		}}
	\end{minipage}
	\hfill
	\begin{minipage}[t]{0.48\textwidth}
		\centering
		\textbf{Pluriel}
		\vspace{0.5em}
		\framebox{\parbox{0.9\linewidth}{\centering
				\vspace{0.5cm}
				\includegraphics[width=0.9\linewidth,clip, trim=0 5cm 0 5cm ]{vokab_300dpi/metschen.png} \\
				\small\textit{Mëtschen (Viennoiseries)}
				\vspace{0.5cm}
		}}
	\end{minipage}
	

	\subsection*{C. Le Livre (\textbf{D'Buch})}
	\begin{tabular}{l l l l}
		\toprule
		\textbf{Forme} & \textbf{Lëtzebuergesch} & \textbf{Français} & \textbf{English} \\
		\midrule
		Singulier Indéfini & \textbf{e Buch} & Un livre & A book \\
		Singulier Défini & \textbf{d'Buch} & Le livre & The book \\
		Pluriel Défini & \textbf{d'Bicher} & Les livres & The books \\
		\bottomrule
	\end{tabular}
	
	\begin{minipage}[t]{0.48\textwidth}
		\centering
		\textbf{Singulier}
		\vspace{0.5em}
		\framebox{\parbox{0.9\linewidth}{\centering
				\vspace{0.5cm}
				\includegraphics[width=0.9\linewidth,clip, trim=0 3cm 0 3cm ]{vokab_300dpi/buch.jpg} \\
				\small\textit{E Buch (Un Livre)}
				\vspace{0.5cm}
		}}
	\end{minipage}
	\hfill
	\begin{minipage}[t]{0.48\textwidth}
		\centering
		\textbf{Pluriel}
		\vspace{0.5em}
		\framebox{\parbox{0.9\linewidth}{\centering
				\vspace{0.5cm}
        \includegraphics[width=0.9\linewidth,clip, trim=0 3cm 0 3cm  ]{vokab_300dpi/bicher.jpg} \\
				\small\textit{Bicher (Livres)}
				\vspace{0.5cm}
		}}
	\end{minipage}
	
	\subsection*{D. La Voiture (\textbf{Den Auto})}
	\begin{tabular}{l l l l}
		\toprule
		\textbf{Forme} & \textbf{Lëtzebuergesch} & \textbf{Français} & \textbf{English} \\
		\midrule
		Singulier Indéfini & \textbf{en Auto} & Une voiture & A car \\
		Singulier Défini & \textbf{den Auto} & La voiture & The car \\
		Pluriel Défini & \textbf{d'Autoen} & Les voitures & The cars \\
		\bottomrule
	\end{tabular}

		\begin{minipage}[t]{0.48\textwidth}
		\centering
		\textbf{Singulier}
		\vspace{0.5em}
		\framebox{\parbox{0.9\linewidth}{\centering
				\vspace{0.5cm}
				\includegraphics[width=0.9\linewidth,clip, trim=0 3cm 0 3cm  ]{vokab_300dpi/auto.jpg} \\
				\small\textit{En Auto (Une Voiture)}
				\vspace{0.5cm}
		}}
	\end{minipage}
	\hfill
	\begin{minipage}[t]{0.48\textwidth}
		\centering
		\textbf{Pluriel}
		\vspace{0.5em}
		\framebox{\parbox{0.9\linewidth}{\centering
				\vspace{0.5cm}
				\includegraphics[width=0.9\linewidth,clip, trim=0 3cm 0 3cm  ]{vokab_300dpi/autoen.jpg} \\
				\small\textit{Autoen (Voitures)}
				\vspace{0.5cm}
		}}
	\end{minipage}

	\subsection*{E. La Peur (\textbf{D'Angscht})}
	\begin{itemize}
		\item \textbf{Ech hunn Angscht virun de Spannen.} $\rightarrow$ J'ai peur des araignées. (I am afraid of spiders.)
		\item \textbf{Keng Angscht, ech si bei dir.} $\rightarrow$ N'aie pas peur, je suis avec toi. (Don't be afraid, I'm with you.)
	\end{itemize}

	\begin{center}
		\begin{minipage}[t]{0.48\textwidth}
			\centering
			\framebox{\parbox{0.9\linewidth}{\centering
					\vspace{0.5cm}
					\includegraphics[width=0.9\linewidth]{vokab_300dpi/angscht.png} \\
					\small\textit{Angscht (La Peur / Fear)}
					\vspace{0.5cm}
			}}
		\end{minipage}
	\end{center}

	\subsection*{F. Le Rhume (\textbf{De Schnapp})}
	\begin{itemize}
		\item \textbf{Hien huet de Schnapp.} $\rightarrow$ Il a un rhume. (He has a cold.)
		\item \textbf{Ech hunn de Schnapp an ech fille mech net gutt.} $\rightarrow$ J'ai un rhume et je ne me sens pas bien. (I have a cold and I don't feel well.)
	\end{itemize}

	\begin{center}
		\begin{minipage}[t]{0.48\textwidth}
			\centering
			\framebox{\parbox{0.9\linewidth}{\centering
					\vspace{0.5cm}
					\includegraphics[width=0.9\linewidth]{vokab_300dpi/schnapp.png} \\
					\small\textit{De Schnapp (Le Rhume / The Cold)}
					\vspace{0.5cm}
			}}
		\end{minipage}
	\end{center}
	
