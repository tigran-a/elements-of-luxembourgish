\chapter{Leçon 2}

\section{Le Verbe \textbf{Kommen} (Venir) - Conjugaison}
\subsection*{A. Présent (Präsens)}
\begin{tabular}{ll c l l}
\toprule
\textbf{Pronom} & \textbf{Lëtzebuergesch} & \textbf{} & \textbf{Français} & \textbf{English} \\
\midrule
 ech & komme(n) & $\rightarrow$ & Je viens & I come \\
 du & kënns & $\rightarrow$ & Tu viens & You come \\
 hie(n) / hatt / et & kënnt & $\rightarrow$ & Il / Elle / On vient & He / She / It comes \\
\midrule
 mir & komme(n) & $\rightarrow$ & Nous venons & We come \\
 dir & kommt & $\rightarrow$ & Vous venez & You come \\
 si & komme(n) & $\rightarrow$ & Ils / Elles viennent & They come \\
\bottomrule
\end{tabular}

\section{Le Verbe \textbf{Goen} (Aller) - Conjugaison}
\subsection*{A. Présent (Präsens)}
\begin{tabular}{ll c l l}
\toprule
\textbf{Pronom} & \textbf{Lëtzebuergesch} & \textbf{} & \textbf{Français} & \textbf{English} \\
\midrule
 ech & gi(nn) & $\rightarrow$ & Je vais & I go \\
 du & gees & $\rightarrow$ & Tu vas & You go \\
 hie(n) / hatt / et & geet & $\rightarrow$ & Il / Elle / On va & He / She / It goes \\
\midrule
 mir & gi(nn) & $\rightarrow$ & Nous allons & We go \\
 dir & gitt & $\rightarrow$ & Vous allez & You go \\
 si & gi(nn) & $\rightarrow$ & Ils / Elles vont & They go \\
\bottomrule
\end{tabular}

\section{Les Mots Interrogatifs (Question Words)}
\begin{itemize}
    \item \textbf{Wat?} $\rightarrow$ Quoi? (What?)
    \item \textbf{Wien?} $\rightarrow$ Qui? (Who?)
    \item \textbf{Wéini?} $\rightarrow$ Quand? (When?)
    \item \textbf{Wou?} / \textbf{Vu wou?} $\rightarrow$ Où? (Where?)
    \item \textbf{Wéi?} $\rightarrow$ Comment? (How?)
    \item \textbf{Firwat?} $\rightarrow$ Pourquoi? (Why?)
\end{itemize}
\begin{itemize}
    \item \textbf{Wéi al bass du?} - Ech si(nn) 90 Joer al. \\ \textit{(Quel âge as-tu? - J'ai 90 ans.)} \\ \textit{(How old are you? - I am 90 years old.)}
    \item \textbf{Wou kënns du hier? / Vu wou kënns du?} - Ech komme(n) aus Syrien. \\ \textit{(D'où viens-tu? - Je viens de Syrie.)} \\ \textit{(Where do you come from? - I come from Syria.)}
    \item \textbf{Wou wunns du?} - Ech wunnen zu Jonglënster. \\ \textit{(Où habites-tu? - J'habite à Junglinster.)} \\ \textit{(Where do you live? - I live in Junglinster.)}
    \item \textbf{Wéini hues du Gebuertsdag?} - Ech hunn haut Gebuertsdag. \\ \textit{(Quand est ton anniversaire? - Mon anniversaire est aujourd'hui.)} \\ \textit{(When is your birthday? - My birthday is today.)}
    \item \textbf{Wéi geet et dir?} - Ech si(nn) midd. \\ \textit{(Comment vas-tu? - Je suis fatigué(e).)} \\ \textit{(How are you? - I am tired.)}
    \item \textbf{Firwat bass du traureg?} - Mäi Bopa ass am Alter vun 90 Joer gestuerwen. \\ \textit{(Pourquoi es-tu triste? - Mon grand-père est décédé à l'âge de 90 ans.)} \\ \textit{(Why are you sad? - My grandfather died at the age of 90.)}
\end{itemize}

\section{Vocabulaire}
\subsection*{A. Adjectifs}
\begin{itemize}
    \item \textbf{al} $\rightarrow$ vieux / âgé (old)
    \item \textbf{nei} $\rightarrow$ nouveau (new)
\end{itemize}
\begin{itemize}
    \item den al\underline{en} Auto / den nei\underline{en} Auto (m)
    \item déi al Saach / déi nei Saach (f)
    \item dat al\underline{t} Haus / dat nei\underline{t} Haus (n)
\end{itemize}

\subsection*{B. Famille}
\begin{itemize}
    \item \textbf{de Mann} $\rightarrow$ le mari (the husband)
    \item \textbf{d'Fra} $\rightarrow$ la femme (the wife)
    \item \textbf{de Jong / de Fils / de Bouf} $\rightarrow$ le fils (the son)
    \item \textbf{d'Duechter / d'Meedchen} $\rightarrow$ la fille (the daughter)
\end{itemize}
\begin{itemize}
    \item De Mann geet spadséieren. (The husband goes for a walk.)
    \item D'Fra geet an den Auchan akafen. (The wife goes shopping at Auchan.)
    \item De Jong geet heem. (The son goes home.)
    \item D'Duechter geet an de Kino. (The daughter goes to the cinema.)
\end{itemize}

\subsection*{C. Meng Famill (My Family)}

\begin{minipage}[t]{0.48\textwidth}
	\centering
	\textbf{Meng Famill}
	\vspace{0.5em}
	\framebox{\parbox{0.9\linewidth}{\centering
			\vspace{0.5cm}
			\includegraphics[width=0.9\linewidth]{vokab_300dpi/famill.png} \\
			\small\textit{Eng grouss Famill (A big family)}
			\vspace{0.5cm}
	}}
\end{minipage}
\hfill
\begin{minipage}[t]{0.48\textwidth}
	\centering
	\textbf{Ech}
	\vspace{0.5em}
	\framebox{\parbox{0.9\linewidth}{\centering
			\vspace{0.5cm}
			\includegraphics[width=0.9\linewidth]{vokab_300dpi/ech.png} \\
			\small\textit{Ech (Moi / Me)}
			\vspace{0.5cm}
	}}
\end{minipage}

\begin{minipage}[t]{0.48\textwidth}
	\centering
	\textbf{Mäi Papp}
	\vspace{0.5em}
	\framebox{\parbox{0.9\linewidth}{\centering
			\vspace{0.5cm}
			\includegraphics[width=0.9\linewidth]{vokab_300dpi/papp.png} \\
			\small\textit{Mäi Papp (Mon père / My father)}
			\vspace{0.5cm}
	}}
\end{minipage}
\hfill
\begin{minipage}[t]{0.48\textwidth}
	\centering
	\textbf{Meng Mamm}
	\vspace{0.5em}
	\framebox{\parbox{0.9\linewidth}{\centering
			\vspace{0.5cm}
			\includegraphics[width=0.9\linewidth]{vokab_300dpi/mamm.png} \\
			\small\textit{Meng Mamm (Ma mère / My mother)}
			\vspace{0.5cm}
	}}
\end{minipage}

\begin{minipage}[t]{0.48\textwidth}
	\centering
	\textbf{Mäi Jong / Bouf / Fils}
	\vspace{0.5em}
	\framebox{\parbox{0.9\linewidth}{\centering
			\vspace{0.5cm}
			\includegraphics[width=0.9\linewidth]{vokab_300dpi/jong.png} \\
			\small\textit{Mäi Jong (Mon fils / My son)}
			\vspace{0.5cm}
	}}
\end{minipage}
\hfill
\begin{minipage}[t]{0.48\textwidth}
	\centering
	\textbf{Meng Duechter / Mäi Meedchen}
	\vspace{0.5em}
	\framebox{\parbox{0.9\linewidth}{\centering
			\vspace{0.5cm}
			\includegraphics[width=0.9\linewidth]{vokab_300dpi/duechter.png} \\
			\small\textit{Meng Duechter (Ma fille / My daughter)}
			\vspace{0.5cm}
	}}
\end{minipage}

\subsection*{D. National Devise}
\begin{itemize}
    \item \textbf{Mir wëlle bleiwe wat mir sinn.} \\ \textit{(Nous voulons rester ce que nous sommes.)} \\ \textit{(We want to remain what we are.)}
\end{itemize}
\parbox{\textwidth}{This is the national motto of Luxembourg. It expresses the Luxembourgish people's desire to maintain their independence and unique cultural identity, especially in the face of foreign influence. The phrase originates from the song "De Feierwon" written by Michel Lentz in 1859.\\ \textit{(Ceci est la devise nationale du Luxembourg. Elle exprime le désir du peuple luxembourgeois de maintenir son indépendance et son identité culturelle unique, notamment face à l'influence étrangère. La phrase provient de la chanson "De Feierwon" écrite par Michel Lentz en 1859.)}}

\subsection*{E. Phrases}
\begin{itemize}
    \item \textbf{Et freet mech / Enchantéiert} \\ \textit{(Enchanté / Enchantée)} \\ \textit{(Pleased to meet you)}
\end{itemize}
