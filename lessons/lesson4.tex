\chapter{Leçon 4}

\section{Greetings and Well-wishes}

\subsection{How are you?}
\begin{itemize}
    \item \textbf{Wéi geet et Iech?} \\
    \textit{(Comment allez-vous? (formel) / How are you? (formal))}
    
    \item \textbf{Wéi geet et dir?} \\
    \textit{(Comment vas-tu? (informel) / How are you? (informal))}
    
    \item \textbf{Wéi ass et?} \\
    \textit{(Comment ça va? / How is it going?)}
\end{itemize}

\subsection{Replies}
\begin{itemize}
    \item \textbf{Gutt, an Iech/dir?} \\
    \textit{(Bien, et vous/toi? / Good, and you?)}
    
    \item \textbf{Ganz gutt, merci.} \\
    \textit{(Très bien, merci. / Very good, thank you.)}
    
    \item \textbf{Et geet.} \\
    \textit{(Ça va. / It's going okay.)}
    
    \item \textbf{Net esou gutt.} \\
    \textit{(Pas très bien. / Not so good.)}

    \item \textbf{Ech si midd.} \\
    \textit{(Je suis fatigué(e). / I am tired.)}
\end{itemize}

\subsection{Well-wishes}
\begin{itemize}
    \item \textbf{Schéinen Dag!} \\
    \textit{(Bonne journée! / Have a nice day!)}
    
    \item \textbf{Schéine Weekend!} \\
    \textit{(Bon week-end! / Have a nice weekend!)}
    
    \item \textbf{Schéine Sonndeg!} \\
    \textit{(Bon dimanche! / Have a nice Sunday!)}

    \item \textbf{Schéinen Nomëtteg!} \\
    \textit{(Bon après-midi! / Have a nice afternoon!)}
\end{itemize}

\section{The Verb \textbf{Wëllen} (To Want) - Conjugation}
\subsection{Présent (Präsens)}
\begin{tabular}{ll c l l}
\toprule
\textbf{Pronom} & \textbf{Lëtzebuergesch} & \textbf{} & \textbf{Français} & \textbf{English} \\
\midrule
 ech & wëll & $\rightarrow$ & Je veux & I want \\
 du & wëlls & $\rightarrow$ & Tu veux & You want \\
 hie(n) / hatt / et & wëll & $\rightarrow$ & Il / Elle / On veut & He / She / It wants \\
\midrule
 mir & wëllen & $\rightarrow$ & Nous voulons & We want \\
 dir & wëllt & $\rightarrow$ & Vous voulez & You want \\
 si & wëllen & $\rightarrow$ & Ils / Elles veulent & They want \\
\bottomrule
\end{tabular}

\section{Examples with \textbf{Wëllen}}
\begin{itemize}
    \item \textbf{Ech wëll e Glas Wäin.} \\
    \textit{(Je veux un verre de vin. / I want a glass of wine.)}
    
    \item \textbf{Wëlls du mat mir an de Kino goen?} \\
    \textit{(Veux-tu aller au cinéma avec moi? / Do you want to go to the cinema with me?)}
    
    \item \textbf{Mir wëlle heem goen.} \\
    \textit{(Nous voulons rentrer à la maison. / We want to go home.)}
\end{itemize}

\section{The words \textbf{Wëll} and \textbf{Well}}
Il est important de distinguer entre "wëll" et "well", qui sont prononcés différemment et ont des significations différentes.
It is important to distinguish between "wëll" and "well", which are pronounced differently and have different meanings.

\subsection{Wëll (from the verb Wëllen)}
This is the first and third person singular conjugation of the verb "wëllen" (to want), as seen in the sections above.

\subsection{Wëll (Wild)}
"Wëll" peut aussi être un adjectif signifiant "sauvage".
"Wëll" can also be an adjective meaning "wild".
\begin{itemize}
    \item \textbf{De Wollef ass e wëllt Déier.} \\
    \textit{(Le loup est un animal sauvage.)}
    \\
    \textit{(The wolf is a wild animal.)}
\end{itemize}

\subsection{Well (Because)}
Le mot "well" est une conjonction signifiant "parce que". C'est un mot très courant en luxembourgeois. Lorsque "well" est utilisé pour relier deux propositions, le verbe de la proposition subordonnée (la partie après "well") est envoyé à la fin de la proposition. C'est une caractéristique essentielle de la grammaire luxembourgeoise.
The word "well" is a conjunction meaning "because". This is a very common word in Luxembourgish. When "well" is used to connect two clauses, the verb in the subordinate clause (the part after "well") is sent to the end of the clause. This is a key feature of Luxembourgish grammar.

\begin{itemize}
    \item \textbf{Ech ginn heem, well ech midd sinn.} \\
    \textit{(Je rentre à la maison parce que je suis fatigué(e).)}
    \\
    \textit{(I'm going home because I am tired.)}
    
    \item \textbf{Hie keeft en Auto, well säin alen Auto futti ass.} \\
    \textit{(Il achète une voiture parce que son ancienne voiture est cassée.)} \\
    \textit{(He is buying a car because his old car is broken.)}
\end{itemize}

\section{The words \textbf{Wann} and \textbf{Dann}}
"Wann" (quand/si) est une conjonction de subordination qui introduit une condition ou un moment dans le temps. "Dann" (alors) est un adverbe qui indique une conséquence.
"Wann" (when/if) is a subordinating conjunction that introduces a condition or a moment in time. "Dann" (then) is an adverb that indicates a consequence.

When a sentence starts with a "wann" clause, the verb of that clause goes to the end. The main clause that follows often starts with its verb, and "dann" can be placed after the verb or omitted entirely, especially in spoken language.

The structure is: \textbf{Wann} [subject] [other elements] \textbf{[verb]}, (\textbf{dann}) \textbf{[verb]} [subject] [other elements].

\subsection{Examples}
\begin{itemize}
    \item \textbf{Wann ech Zäit hunn, (dann) ginn ech an de Kino.} \\
    \textit{(Quand j'ai le temps, (alors) je vais au cinéma.)} \\
    \textit{(When I have time, (then) I go to the cinema.)}
    
    \item \textbf{Wann d'Sonn schéngt, (dann) si mir frou.} \\
    \textit{(Quand le soleil brille, (alors) nous sommes contents.)} \\
    \textit{(When the sun shines, (then) we are happy.)}

    \item \textbf{Wann s du wëlls, (dann) kënne mir eppes iessen goen.} \\
    \textit{(Si tu veux, (alors) nous pouvons aller manger quelque chose.)} \\
    \textit{(If you want, (then) we can go eat something.)}
\end{itemize}

\section{The verbs \textbf{Brauchen} and \textbf{Mussen}}

\subsection{Conjugation of \textbf{Brauchen} (to need)}
\begin{tabular}{ll c l l}
\toprule
\textbf{Pronom} & \textbf{Lëtzebuergesch} & \textbf{} & \textbf{Français} & \textbf{English} \\
\midrule
 ech & brauch & $\rightarrow$ & J'ai besoin de & I need \\
 du & brauchs & $\rightarrow$ & Tu as besoin de & You need \\
 hie(n) / hatt / et & brauch & $\rightarrow$ & Il / Elle / On a besoin de & He / She / It needs \\
\midrule
 mir & brauchen & $\rightarrow$ & Nous avons besoin de & We need \\
 dir & braucht & $\rightarrow$ & Vous avez besoin de & You need \\
 si & brauchen & $\rightarrow$ & Ils / Elles ont besoin de & They need \\
\bottomrule
\end{tabular}

\subsection{Conjugation of \textbf{Mussen} (to have to/must)}
\begin{tabular}{ll c l l}
\toprule
\textbf{Pronom} & \textbf{Lëtzebuergesch} & \textbf{} & \textbf{Français} & \textbf{English} \\
\midrule
 ech & muss & $\rightarrow$ & Je dois & I must \\
 du & muss & $\rightarrow$ & Tu dois & You must \\
 hie(n) / hatt / et & muss & $\rightarrow$ & Il / Elle / On doit & He / She / It must \\
\midrule
 mir & mussen & $\rightarrow$ & Nous devons & We must \\
 dir & musst & $\rightarrow$ & Vous devez & You must \\
 si & mussen & $\rightarrow$ & Ils / Elles doivent & They must \\
\bottomrule
\end{tabular}

\subsection{Difference and Examples}
\textbf{Brauchen} est utilisé pour exprimer le besoin de quelque chose.
\textbf{Brauchen} is used to express a need for something.
\begin{itemize}
    \item \textbf{Ech brauch e neien Auto.} (J'ai besoin d'une nouvelle voiture. / I need a new car.)
    \item \textbf{Mir brauche méi Zäit.} (Nous avons besoin de plus de temps. / We need more time.)
\end{itemize}
\textbf{Mussen} est utilisé pour exprimer une obligation ou une nécessité de faire quelque chose.
\textbf{Mussen} is used to express an obligation or necessity to do something.
\begin{itemize}
    \item \textbf{Ech muss heem goen.} (Je dois rentrer à la maison. / I must go home.)
    \item \textbf{Si muss schaffen.} (Elle doit travailler. / She has to work.)
\end{itemize}

\subsection{Negation}
La négation de ces verbes est un point de confusion courant.
The negation of these verbs is a common point of confusion.
\begin{itemize}
    \item \textbf{Net brauchen} signifie "ne pas avoir besoin de". Il exprime une absence de nécessité. \\
    \textbf{Net brauchen} means "not to need" or "not to have to". It expresses a lack of necessity. \\
    \textbf{Example:} \textit{Du brauchs net ze waarden.} (Tu n'as pas besoin d'attendre. / You don't need to wait.)
    \item \textbf{Net mussen} signifie également "ne pas devoir" ou "ne pas être obligé de". \\
    \textbf{Net mussen} also means "not to have to". \\
    \textbf{Example:} \textit{Du muss net kommen.} (Tu n'es pas obligé de venir. / You don't have to come.)
    \item Pour exprimer une interdiction ("must not"), on utilise généralement le verbe \textbf{däerfen} (pouvoir/être autorisé) à la forme négative. \\
    To express a prohibition ("must not"), Luxembourgers usually use the verb \textbf{däerfen} (to be allowed to) in the negative. \\
    \textbf{Example:} \textit{Du däerfs hei net fëmmen.} (Tu ne dois pas fumer ici. / You must not smoke here.)
\end{itemize}

\section{Time-related words}

\subsection{Adverbs}
\begin{itemize}
    \item \textbf{Elo} (maintenant / now) \\
    \textit{Elo ginn ech iessen.} (Maintenant, je vais manger. / Now I'm going to eat.)
    \item \textbf{Virdrun} (avant / before that) \\
    \textit{Ech ginn an de Kino, mee virdrun iessen ech eppes.} (Je vais au cinéma, mais avant cela je mange quelque chose. / I'm going to the cinema, but before that I'll eat something.)
    \item \textbf{Duerno} (après / afterwards) \\
    \textit{Mir ginn iessen, an duerno an de Kino.} (Nous allons manger, et après au cinéma. / We're going to eat, and afterwards to the cinema.)
\end{itemize}

\subsection{Prepositions}
\begin{itemize}
    \item \textbf{Virun / Virum} (avant / before) \\
    \textit{Virum} est la contraction de \textit{virun} + \textit{dem}. \\
    \textit{Virum} is the contraction of \textit{virun} + \textit{dem}. \\
    \textbf{Example:} \textit{Mir treffen eis virum Iessen.} (Nous nous retrouvons avant le repas. / We'll meet before the meal.)
    \item \textbf{No / Nom} (après / after) \\
    \textit{Nom} est la contraction de \textit{no} + \textit{dem}. \\
    \textit{Nom} is the contraction of \textit{no} + \textit{dem}. \\
    \textbf{Example:} \textit{Nom Iessen drénke mir e Patt.} (Après le repas, nous prenons un verre. / After the meal, we'll have a drink.)
\end{itemize}

\subsection{Conjunctions}
\begin{itemize}
    \item \textbf{Nodeems} (après que / after) \\
    \textit{Nodeems} est une conjonction de subordination, ce qui signifie que le verbe est placé à la fin de la proposition. \\
    \textit{Nodeems} is a subordinating conjunction, which means the verb goes to the end of the clause. \\
    \textbf{Example:} \textit{Nodeems ech giess hunn, ginn ech an de Kino.} (Après que j'ai mangé, je vais au cinéma. / After I have eaten, I go to the cinema.)
    \item \textbf{Ier / Éier} (avant que / before) \\
    \textit{Ier} ou \textit{Éier} est une conjonction de subordination, ce qui signifie que le verbe est placé à la fin de la proposition. \\
    \textit{Ier} or \textit{Éier} is a subordinating conjunction, which means the verb goes to the end of the clause. \\
    \textbf{Example:} \textit{Ier ech an d'Bett ginn, liesen ech e Buch.} (Avant que j'aille au lit, je lis un livre. / Before I go to bed, I read a book.)
\end{itemize}

\section{The song "Kättche, Kättche"}
"Kättche, Kättche" est une chanson traditionnelle luxembourgeoise, célébrant le vin local de la Moselle et souvent chantée lors d'occasions festives. Elle tisse magnifiquement les thèmes de l'identité nationale, notamment avec une citation directe de la devise nationale.
"Kättche, Kättche" is a traditional Luxembourgish song, celebrating the local Moselle wine and often sung at festive occasions. It beautifully weaves in themes of national identity, notably with a direct quote from the national motto.

\begin{itemize}
    \item \textbf{Et wuessen an de frieme Länner,} \\
    \textit{(Il pousse dans les pays étrangers,) / (In foreign lands there grow,)}
    \item \textbf{Vill schwéier Wäiner rout a wäiss.} \\
    \textit{(Beaucoup de vins lourds rouges et blancs.) / (Many heavy wines, red and white.)}
    \item \textbf{Si si gesicht vu ville Kenner,} \\
    \textit{(Ils sont recherchés par de nombreux connaisseurs,) / (They are sought by many connoisseurs,)}
    \item \textbf{Vum Rhäin bis déi Säit vu Paräis.} \\
    \textit{(Du Rhin jusqu'au-delà de Paris.) / (From the Rhine to beyond Paris.)}
    \item \textbf{Ech hale mech un d’Muselblimmchen,} \\
    \textit{(Moi, je m'en tiens à la petite fleur de la Moselle,) / (I stick to the little Moselle flower,)}
    \item \textbf{Dat ass de Wéngche fir eist d’Land.} \\
    \textit{(C'est le petit vin pour notre pays.) / (That is the little wine for our land.)}
    \item \textbf{E geet dem Jonktem wéi dem ‘Eimschen,} \\
    \textit{(Il convient aussi bien au jeune qu'au vieux,) / (It suits the young as well as the old,)}
    \item \textbf{Wann si en drénke mat Verstand.} \\
    \textit{(Quand ils le boivent avec modération.) / (When they drink it with understanding.)}
\end{itemize}

\textbf{Refrain: (2 mol). Kättche, Kättche, bréng mer nach e Pättchen,} \\
\textit{(Refrain: (2 fois). Catherine, Catherine, apporte-moi encore un petit pot,) / (Refrain: (2 times). Katie, Katie, bring me another little pot,)}
\textbf{Vun der Musel a soss keen.} \\
\textit{(De la Moselle et personne d'autre.) / (From the Moselle and none other.)}
\textbf{Ei, wéi schmaacht mer dee Kadettchen,} \\
\textit{(Oh, comme ce petit Kadett me plaît,) / (Oh, how that little Kadett tastes good to me,)}
\textbf{‘t as en Dronk fir Broscht a Been:} \\
\textit{(C'est une boisson pour la poitrine et les jambes:) / (It's a drink for chest and legs:)}
\textbf{(2 mol)} \\
\textit{((2 fois)) / ((2 times)) }

\begin{itemize}
    \item \textbf{Op dir bedréift sidd oder lëschteg,} \\
    \textit{(Que vous soyez triste ou joyeux,) / (Whether you are sad or merry,)}
    \item \textbf{En deet séng Flicht zu jidder Stonn.} \\
    \textit{(Il fait son devoir à toute heure.) / (It does its duty at every hour.)}
    \item \textbf{En as vun heem aus fromm a krëschtlech,} \\
    \textit{(Il est d'origine pieux et chrétien,) / (It is by nature pious and Christian,)}
    \item \textbf{E krut de Seege vun der Sonn,} \\
    \textit{(Il a reçu la bénédiction du soleil,) / (It received the blessing of the sun,)}
    \item \textbf{Bleiwt him ewesch mat Zockerwaasser,} \\
    \textit{(Restez loin de lui avec de l'eau sucrée,) / (Stay away from it with sugar water,)}
    \item \textbf{E brauch nët méi gedeeft ze gin.} \\
    \textit{(Il n'a plus besoin d'être baptisé.) / (It no longer needs to be baptized.)}
    \item \textbf{E séngt mam Lentz nët fir ze spaassen:} \\
    \textit{(Il ne chante pas avec Lentz pour plaisanter:) / (It doesn't sing with Lentz for fun:)}
    \item \textbf{‘‘ Mir wëlle bleiwen wat mir sin. ‘‘} \\
    \textit{(« Nous voulons rester ce que nous sommes. ») / (‘‘ We want to remain what we are. ’’)}
\end{itemize}

\textbf{Refrain: (2 mol). Kättche, Kättche, ,,,,.} \\
\textit{(Refrain: (2 fois). Catherine, Catherine, ,,,.) / (Refrain: (2 times). Katie, Katie, ,,,.)}

\section{Vocabulaire (Vocabulary)}

\begin{minipage}[t]{0.48\textwidth}
    \centering
    \textbf{Dobaussen}
    \vspace{0.5em}
    \framebox{\parbox{0.9\linewidth}{\centering
            \vspace{0.5cm}
            \includegraphics[width=0.9\linewidth]{vokab_300dpi/dobaussen.png} \\
            \small\textit{Dobaussen (Dehors / Outside)}
            \vspace{0.5cm}
    }}
\end{minipage}
\hfill
\begin{minipage}[t]{0.48\textwidth}
    \centering
    \textbf{Dobannen}
    \vspace{0.5em}
    \framebox{\parbox{0.9\linewidth}{\centering
            \vspace{0.5cm}
            \includegraphics[width=0.9\linewidth]{vokab_300dpi/dobannen.png} \\
            \small\textit{Dobannen (Dedans / Inside)}
            \vspace{0.5cm}
    }}
\end{minipage}
\hfill
\begin{minipage}[t]{0.48\textwidth}
    \centering
    \textbf{Kalt}
    \vspace{0.5em}
    \framebox{\parbox{0.9\linewidth}{\centering
            \vspace{.5cm}
            \includegraphics[width=0.9\linewidth]{vokab_300dpi/kalt.png} \\
            \small\textit{Kalt (Froid / Cold)}
            \vspace{0.5cm}
    }}
\end{minipage}
\hfill
\begin{minipage}[t]{0.48\textwidth}
    \centering
    \textbf{Waarm}
    \vspace{0.5em}
    \framebox{\parbox{0.9\linewidth}{\centering
            \vspace{0.5cm}
            \includegraphics[width=0.9\linewidth]{vokab_300dpi/waarm.png} \\
            \small\textit{Waarm (Chaud / Warm)}
            \vspace{0.5cm}
    }}
\end{minipage}
\hfill
\begin{minipage}[t]{0.48\textwidth}
    \centering
    \textbf{Eng Tass}
    \vspace{0.5em}
    \framebox{\parbox{0.9\linewidth}{\centering
            \vspace{0.5cm}
            \includegraphics[width=0.9\linewidth]{vokab_300dpi/Tass.png} \\
            \small\textit{Tass (Tasse / Cup)}
            \vspace{0.5cm}
    }}
\end{minipage}
\hfill
\begin{minipage}[t]{0.48\textwidth}
    \centering
    \textbf{E Läffel}
    \vspace{0.5em}
    \framebox{\parbox{0.9\linewidth}{\centering
            \vspace{0.5cm}
            \includegraphics[width=0.9\linewidth]{vokab_300dpi/Läffel.png} \\
            \small\textit{Läffel (Cuillère / Spoon)}
            \vspace{0.5cm}
    }}
\end{minipage}
\hfill
\begin{minipage}[t]{0.48\textwidth}
    \centering
    \textbf{Eng Fläsch}
    \vspace{0.5em}
    \framebox{\parbox{0.9\linewidth}{\centering
            \vspace{0.5cm}
            \includegraphics[width=0.9\linewidth]{vokab_300dpi/Fläsch.png} \\
            \small\textit{Fläsch (Bouteille / Bottle)}
            \vspace{0.5cm}
    }}
\end{minipage}
\hfill
\begin{minipage}[t]{0.48\textwidth}
    \centering
    \textbf{Eng Täsch}
    \vspace{0.5em}
    \framebox{\parbox{0.9\linewidth}{\centering
            \vspace{0.5cm}
            \includegraphics[width=0.9\linewidth]{vokab_300dpi/Täsch.png} \\
            \small\textit{Täsch (Sac / Bag)}
            \vspace{0.5cm}
    }}
\end{minipage}
\hfill
\begin{minipage}[t]{0.48\textwidth}
    \centering
    \textbf{En Handy}
    \vspace{0.5em}
    \framebox{\parbox{0.9\linewidth}{\centering
            \vspace{0.5cm}
            \includegraphics[width=0.9\linewidth]{vokab_300dpi/Handy.png} \\
            \small\textit{Handy (Portable / Mobile phone)}
            \vspace{0.5cm}
    }}
\end{minipage}
\hfill
\begin{minipage}[t]{0.48\textwidth}
    \centering
    \textbf{E Glas Wäin}
    \vspace{0.5em}
    \framebox{\parbox{0.9\linewidth}{\centering
            \vspace{0.5cm}
            \includegraphics[width=0.9\linewidth]{vokab_300dpi/glas_wäin.png} \\
            \small\textit{E Glas Wäin (Un verre de vin / A glass of wine)}
            \vspace{0.5cm}
    }}
\end{minipage}
