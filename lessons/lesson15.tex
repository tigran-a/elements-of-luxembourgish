\chapter{Leçon 15}

\section{Affirmativ, Interrogativ \& Negativ}

An dëser Lektioun kucke mir, wéi ee Froen stellt an negativ Äntwerte gëtt.

\subsection*{Sazbau / Structure des phrases}

\begin{itemize}
    \item \textbf{Affirmativ:} Ech hunn e Mupp. (I have a dog. / J'ai un chien.)
    \item \textbf{Interrogativ:} Hutt Dir / hues du e Mupp? (Do you have a dog? / Avez-vous un chien ? / As-tu un chien ?)
    \item \textbf{Negativ:} Nee, ech hu kee Mupp. (No, I don't have a dog. / Non, je n'ai pas de chien.)
    \item \textbf{Affirmativ Äntwert:} Jo, ech hunn e Mupp. (Yes, I have a dog. / Oui, j'ai un chien.)
\end{itemize}

\subsection*{Plural Beispill}

\begin{itemize}
    \item \textbf{Affirmativ:} Ech hu zwou Kazen. (I have two cats. / J'ai deux chats.)
    \item \textbf{Interrogativ:} Hutt Dir och en Auto? (Do you also have a car? / Avez-vous aussi une voiture ?)
    \item \textbf{Äntwert:} Ech hunn e ganz alen Auto. (I have a very old car. / J'ai une très vieille voiture.)
\end{itemize}

\section{Erklärungen: Keen vs Keng}

In Luxembourgish, the negative article "no" (not any) changes according to the gender and number of the noun.

\begin{itemize}
    \item \textbf{Keen}: Used for masculine and neuter singular nouns.
    \begin{itemize}
        \item \textit{Ech hu \textbf{kee} Mupp.} (Masc.)
        \item \textit{Ech hu \textbf{kee} Kand.} (Neut.)
    \end{itemize}
    \item \textbf{Keng}: Used for feminine singular and all plural nouns.
    \begin{itemize}
        \item \textit{Ech hu \textbf{keng} Kaz.} (Fem.)
        \item \textit{Ech hu \textbf{keng} Kanner.} (Plural)
    \end{itemize}
\end{itemize}

\section{Velo fueren (Cycling)}

If you haven't done something for a long time, we use the structure ``scho laang net méi''.

\begin{itemize}
    \item \textbf{Ech si scho laang net méi Velo gefuer.} \\
    (I haven't cycled for a long time. / Je n'ai pas fait de vélo depuis longtemps.)
    \item \textbf{Ech hunn e bëssen Angscht, Velo ze fueren, well ech et scho laang net méi gemaach hunn.} \\
    (I'm a bit afraid to cycle because I haven't done it for a long time. / J'ai un peu peur de faire du vélo car je n'en ai pas fait depuis longtemps.)
\end{itemize}

\section{D'Famill}

\begin{itemize}
    \item \textbf{Hutt Dir eng Famill?} \\
    (Do you have a family? / Avez-vous une famille ?)
    \item \textbf{Jo, ech hu Mann, ee Jong an eng Duechter.} \\
    (Yes, I have a husband, a son and a daughter. / Oui, j'ai un mari, un fils et une fille.)
    \item \textbf{Jo, mir hunn eng grouss Famill mat villen Kosengen a Kusinnen.} \\
    (Yes, we have a large family with many cousins. / Oui, nous avons une grande famille avec beaucoup de cousins et couines.)
\end{itemize}

\section{Iessen \& Juegd}

\begin{itemize}
    \item \textbf{Ech iesse gär, am léifsten Fleesch.} \\
    (I like to eat, preferably meat. / J'aime manger, de préférence de la viande.)
    \item \textbf{Bass du e Jeeër?} \\
    (Are you a hunter? / Es-tu un chasseur ?)
    \item \textbf{Gees du op d'Juegd?} \\
    (Do you go hunting? / Vas-tu à la chasse ?)
    \item \textbf{E blesséiert Déier.} \\
    (An injured animal. / Un animal blessé.)
\end{itemize}

\section{Wëll Déieren (Wild Animals)}

\begin{itemize}
    \item \textbf{E Fuuss / Fiiss} (Fox / Foxes / Renard)
    \item \textbf{E Réi / Réi} (Roe deer / Chevreuil)
    \item \textbf{E Hirsch / Hirschen} (Red deer / Cerf)
    \item \textbf{E Wëllschwäin / Wëllschwäin} (Wild boar / Sanglier) \\
    \textit{Variant: Wëllt Schwäin / Wëll Schwäin}
    \item \textbf{E Hues / Huesen} (Hare / Lièvre) \\
    \textit{Note: "eng Hues" (plural: Huesen) is a type of short sock or slipper (chausson), not to be confused with the animal "e Hues".}
\end{itemize}

\section{Divers}

\begin{itemize}
    \item \textbf{Séissegkeeten} (Sweets / Sucreries)
    \item \textbf{Kamellen} (Candies / Bonbons)
    \item \textbf{Kichelcher} (Cookies / Biscuits)
    \item \textbf{Schockela} (Chocolate / Chocolat)
    \item \textbf{e Lutscher} (Lollipop / Sucette)
    \item \textbf{Knätsch} (Chewing gum / Chewing-gum)
    \item \textbf{eng Glace} (Ice cream / Glace)
\end{itemize}

\section{MOUD - KLEEDER (Fashion \& Clothing / Mode \& Vêtements)}

\subsection*{Froen an Äntwerten (Questions and Answers / Questions et Réponses)}

\begin{itemize}
    \item \textbf{Wéi wichteg ass d'Moud fir Iech?} \\
    \textit{(How important is fashion to You? / Quelle est l'importance de la mode pour Vous ?)}
    \begin{itemize}
        \item Ganz wichteg, ech wëll ëmmer gutt ausgesinn. \\
        \small{(Very important, I always want to look good. / Très important, je veux toujours avoir une bonne apparence.)}
        \item Net sou wichteg, ech droe léiwer gemittlech Kleeder. \\
        \small{(Not so important, I prefer wearing comfortable clothes. / Pas très important, je préfère porter des vêtements confortables.)}
        \item Et kënnt drop un, fir d'Aarbecht ass et wichteg. \\
        \small{(It depends, it's important for work. / Ça dépend, c'est important pour le travail.)}
    \end{itemize}

    \item \textbf{Wéi informéiert Dir Iech iwwer d'Moud?} \\
    \textit{(How do You stay informed about fashion? / Comment Vous informez-vous sur la mode ?)}
    \begin{itemize}
        \item Ech liesen Zäitschrëften wéi ``Vogue''. \\
        \small{(I read magazines like "Vogue". / Je lis des magazines comme « Vogue ».)}
        \item Ech kucken op Instagram an anere sozialen Medien. \\
        \small{(I look at Instagram and other social media. / Je regarde sur Instagram et d'autres réseaux sociaux.)}
        \item Ech kucken einfach an d'Schaufenster vun de Geschäfter. \\
        \small{(I just look at shop windows. / Je regarde simplement les vitrines des magasins.)}
    \end{itemize}

    \item \textbf{Kennt Dir Zäitschrëften iwwer d'Moud?} \\
    \textit{(Do You know any fashion magazines? / Connaissez-vous des magazines de mode ?)}
    \begin{itemize}
        \item Jo, ``Elle'' an ``Vogue'' sinn déi bekanntst. \\
        \small{(Yes, "Elle" and "Vogue" are the most famous. / Oui, « Elle » et « Vogue » sont les plus connus.)}
        \item Ech kennen och ``L'Officiel''. \\
        \small{(I also know "L'Officiel". / Je connais aussi « L'Officiel ».)}
    \end{itemize}

    \item \textbf{Wat ass Äre Gout?} \\
    \textit{(What is Your taste? / Quel est Votre goût ?)}
    \begin{itemize}
        \item Mäi Gout ass éischter klassesch. \\
        \small{(My taste is rather classic. / Mon goût est plutôt classique.)}
        \item Ech hu léiwer e sportleche Stil. \\
        \small{(I prefer a sporty style. / Je préfère un style sportif.)}
        \item Ech hunn et gär modern an elegant. \\
        \small{(I like it modern and elegant. / J'aime le style moderne et élégant.)}
    \end{itemize}

    \item \textbf{Kaaft Dir gär Kleeder?} \\
    \textit{(Do You like buying clothes? / Aimez-vous acheter des vêtements ?)}
    \begin{itemize}
        \item Jo, ech ginn immens gär akafen. \\
        \small{(Yes, I love going shopping. / Oui, j'adore faire du shopping.)}
        \item Nee, et ass éischter eng Plo fir mech. \\
        \small{(No, it's more of a chore for me. / Non, c'est plutôt une corvée pour moi.)}
        \item Nëmme wann ech wierklech eppes brauch. \\
        \small{(Only when I really need something. / Seulement quand j'ai vraiment besoin de quelque chose.)}
    \end{itemize}

    \item \textbf{Wéi dacks gitt Dir Kleeder akafen?} \\
    \textit{(How often do You go clothes shopping? / À quelle fréquence allez-vous acheter des vêtements ?)}
    \begin{itemize}
        \item Eemol am Mount. \\
        \small{(Once a month. / Une fois par mois.)}
        \item Virun all neier Saison. \\
        \small{(Before every new season. / Avant chaque nouvelle saison.)}
        \item Ganz seelen, vläicht zweemol am Joer. \\
        \small{(Very rarely, maybe twice a year. / Très rarement, peut-être deux fois par an.)}
    \end{itemize}

    \item \textbf{Wou kaaft Dir d'Kleeder?} \\
    \textit{(Where do You buy Your clothes? / Où achetez-vous vos vêtements ?)}
    \begin{itemize}
        \item Am Zentrum vun der Stad. \\
        \small{(In the city center. / Au centre-ville.)}
        \item An der Belle Étoile oder am City Concorde. \\
        \small{(In Belle Étoile or City Concorde. / À la Belle Étoile ou au City Concorde.)}
        \item An engem klenge Boutique. \\
        \small{(In a small boutique. / Dans une petite boutique.)}
    \end{itemize}

    \item \textbf{Kaaft Dir och Kleeder am Ausland?} \\
    \textit{(Do You also buy clothes abroad? / Achetez-vous aussi des vêtements à l'étranger ?)}
    \begin{itemize}
        \item Jo, oft an der Belsch oder a Frankräich. \\
        \small{(Yes, often in Belgium or France. / Oui, souvent en Belgique ou en France.)}
        \item Heiansdo wann ech an der vakanz sinn. \\
        \small{(Sometimes when I'm on vacation. / Parfois quand je suis en vacances.)}
        \item Nee, ech kafen alles hei zu Lëtzebuerg. \\
        \small{(No, I buy everything here in Luxembourg. / Non, j'achète tout ici au Luxembourg.)}
    \end{itemize}

    \item \textbf{Kaaft Dir Kleeder um Internet?} \\
    \textit{(Do You buy clothes on the internet? / Achetez-vous des vêtements sur Internet ?)}
    \begin{itemize}
        \item Jo, dat ass ganz praktesch. \\
        \small{(Yes, that's very practical. / Oui, c'est très pratique.)}
        \item Heiansdo, awer ech probéieren d'Kleeder léiwer un. \\
        \small{(Sometimes, but I prefer to try the clothes on. / Parfois, mais je préfère essayer les vêtements.)}
        \item Nee, ech wëll d'Kleeder gesinn an uprobéieren. \\
        \small{(No, I want to see and try on the clothes. / Non, je veux voir et essayer les vêtements.)}
    \end{itemize}

    \item \textbf{Wat fir Kleeder ditt Dir un, fir schaffen ze goen?} \\
    \textit{(What clothes do You put on to go to work? / Quels vêtements mettez-vous pour aller travailler ?)}
    \begin{itemize}
        \item E Kostüm an eng Hiem. \\
        \small{(A suit and a shirt. / Un costume et une chemise.)}
        \item Eng Jeans an e Pullover. \\
        \small{(Jeans and a sweater. / Un jean et un pull.)}
        \item E Rack oder eng Jupe. \\
        \small{(A dress or a skirt. / Une robe ou une jupe.)}
    \end{itemize}

    \item \textbf{Wat ditt Dir doheem un?} \\
    \textit{(What do You wear at home? / Que portez-vous à la maison ?)}
    \begin{itemize}
        \item E Jogging oder eng gemittlech Box. \\
        \small{(A tracksuit or comfortable pants. / Un jogging ou un pantalon confortable.)}
        \item Einfach en T-Shirt a Shorts. \\
        \small{(Just a T-shirt and shorts. / Juste un T-shirt et un short.)}
    \end{itemize}

    \item \textbf{Wat ditt Dir un, fir op eng Hochzäit ze goen?} \\
    \textit{(What do You wear to go to a wedding? / Que portez-vous pour aller à un mariage ?)}
    \begin{itemize}
        \item En elegante Rack. \\
        \small{(An elegant dress. / Une robe élégante.)}
        \item E Kostüm mat enger Cravate. \\
        \small{(A suit with a tie. / Un costume avec une cravate.)}
    \end{itemize}

    \item \textbf{Wat ditt Dir un, fir op eng Grillparty ze goen?} \\
    \textit{(What do You wear to go to a barbecue party? / Que portez-vous pour aller à une grillade ?)}
    \begin{itemize}
        \item Eng Shorts an en T-Shirt. \\
        \small{(Shorts and a T-shirt. / Un short et un T-shirt.)}
        \item Eng Summerkleed. \\
        \small{(A summer dress. / Une robe d'été.)}
    \end{itemize}
\end{itemize}

\section{Illustréiert Wierder / Illustrated Words}

\begin{multicols}{2}

\begin{minipage}[t]{0.48\textwidth}
    \centering
    \textbf{e Fuuss}
    \vspace{0.5em}
    \framebox{\parbox{0.9\linewidth}{\centering
            \vspace{0.5cm}
            \includegraphics[width=0.9\linewidth]{vokab_300dpi/fuuss.png} \\
            \small\textit{Renard / Fox} \\
            \textit{Pl: Fiiss}
            \vspace{0.5cm}
    }}
\end{minipage}
\hfill
\begin{minipage}[t]{0.48\textwidth}
    \centering
    \textbf{e Réi}
    \vspace{0.5em}
    \framebox{\parbox{0.9\linewidth}{\centering
            \vspace{0.5cm}
            \includegraphics[width=0.9\linewidth]{vokab_300dpi/rei.png} \\
            \small\textit{Chevreuil / Roe deer} \\
            \textit{Pl: Réi}
            \vspace{0.5cm}
    }}
\end{minipage}

\vspace{0.5cm}

\begin{minipage}[t]{0.48\textwidth}
    \centering
    \textbf{e Hirsch}
    \vspace{0.5em}
    \framebox{\parbox{0.9\linewidth}{\centering
            \vspace{0.5cm}
            \includegraphics[width=0.9\linewidth]{vokab_300dpi/hirsch.png} \\
            \small\textit{Cerf / Red deer} \\
            \textit{Pl: Hirschen}
            \vspace{0.5cm}
    }}
\end{minipage}
\hfill
\begin{minipage}[t]{0.48\textwidth}
    \centering
    \textbf{e Wëllschwäin}
    \vspace{0.5em}
    \framebox{\parbox{0.9\linewidth}{\centering
            \vspace{0.5cm}
            \includegraphics[width=0.9\linewidth]{vokab_300dpi/wellschwaein.png} \\
            \small\textit{Sanglier / Wild boar} \\
            \textit{Pl: Wëllschwäin}
            \vspace{0.5cm}
    }}
\end{minipage}

\vspace{0.5cm}

\begin{minipage}[t]{0.48\textwidth}
    \centering
    \textbf{e Hues}
    \vspace{0.5em}
    \framebox{\parbox{0.9\linewidth}{\centering
            \vspace{0.5cm}
            \includegraphics[width=0.9\linewidth]{vokab_300dpi/hues.png} \\
            \small\textit{Lièvre / Hare} \\
            \textit{Pl: Huesen}
            \vspace{0.5cm}
    }}
\end{minipage}

\end{multicols}
