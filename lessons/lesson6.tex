\chapter{Leçon 6}

\section{Niklosdag}
La Saint-Nicolas, ou Niklosdag, est une tradition très populaire au Luxembourg, célébrée le 6 décembre. C'est une journée dédiée à Saint Nicolas, le saint patron des enfants. La veille du Niklosdag, les enfants placent leurs chaussons devant la porte de leur chambre, espérant que den Niklos (Saint Nicolas) les remplira de bonbons et de cadeaux pendant la nuit. Il est accompagné du "Houseker", un personnage menaçant au visage peint en noir qui porte une cravache pour punir les enfants désobéissants.

Niklosdag, or Saint Nicholas' Day, is a very popular tradition in Luxembourg, celebrated on December 6th. It is a day dedicated to Saint Nicholas, the patron saint of children. On the eve of Niklosdag, children place their slippers in front of their bedroom door, hoping that den Niklos (Saint Nicholas) will fill them with sweets and presents during the night. He is accompanied by the "Houseker", a menacing figure with a black-painted face who carries a whip to punish naughty children.

\section{The Story of St. Nicholas and the Three Boys}
Voici une histoire traditionnelle de Saint Nicolas, bien connue au Luxembourg et dans d'autres régions d'Europe. 

This is a traditional story about Saint Nicholas, which is well-known in Luxembourg and other European regions.

\begin{itemize}
    \item \textbf{Owend virun Niklosdag 3 Bouwen spillen an der Gaass.} \\
    \textit{(La veille de la Saint-Nicolas, 3 garçons jouent dans la ruelle / On the eve of Saint Nicholas' Day, 3 boys are playing in the alley.)}
    \item \textbf{Et ass spéit. Si sollen heemgoen.} \\
    \textit{(Il est tard. Ils devraient rentrer à la maison / It is late. They should go home.)}
    \item \textbf{Awer si spille weider.} \\
    \textit{(Mais ils continuent à jouer / But they keep playing.)}
    \item \textbf{Et gëtt däischter.} \\
    \textit{(Il fait sombre / It is getting dark.)}
    \item \textbf{Si fannen de Wee fir heem net méi.} \\
    \textit{(Ils ne trouvent plus le chemin de la maison / They can't find their way home anymore.)}
    \item \textbf{An engem Haus brennt nach Luucht.} \\
    \textit{(Dans une maison, la lumière est encore allumée / In a house, a light is still burning.)}
    \item \textbf{Die Bouwen ginn eran.} \\
    \textit{(Les garçons entrent / The boys go inside.)}
    \item \textbf{Dobannen schafft e Metzler.} \\
    \textit{(À l'intérieur, un boucher travaille / Inside, a butcher is working.)}
    \item \textbf{Et richt no Jelli.} \\
    \textit{(Ça sent la gelée / It smells like jelly.)}
    \item \textbf{De Metzler hëllt d’Jongen.} \\
    \textit{(Le boucher prend les garçons / The butcher takes the boys.)}
    \item \textbf{Schneid se a Stécker a leet se an Eng Bitchen.} \\
    \textit{(Il les coupe en morceaux et les met dans un tonneau / He cuts them into pieces and puts them in a barrel.)} \\
    \textit{("Bitchen" est un tonneau ou un saloir / "Bitchen" is a barrel or a salting tub.)}
    \item \textbf{E puer Stonnen duerno, klappt et un der Dier.} \\
    \textit{(Quelques heures plus tard, on frappe à la porte / A few hours later, there is a knock on the door.)}
    \item \textbf{Et ass den Zinniklos.} \\
    \textit{(C'est Saint Nicolas / It is Saint Nicholas.)}
    \item \textbf{O, komm eran, du gudde Mann, ech zerwéieren Iech eng Schësselchen gudde Jelli.} \\
    \textit{(Oh, entrez, brave homme, je vous servirai un bol de bonne gelée / Oh, come in, good man, I will serve you a bowl of good jelly.)}
    \item \textbf{O, nee, seet de Niklos, hei richt et no Mënschefleesch.} \\
    \textit{(Oh, non, dit Nicolas, ici ça sent la chair humaine / Oh, no, says Nicholas, it smells like human flesh here.)}
    \item \textbf{Hien geet bäi d’Bitchen, hält säi Staf driwwer an déi dräi Bouwen klammen gesond a monter eraus.} \\
    \textit{(Il va au tonneau, tient son bâton dessus et les trois garçons en sortent sains et vifs / He goes to the barrel, holds his staff over it and the three boys climb out healthy and cheerful.)}
\end{itemize}

\section{Verbs}
\subsection*{Spillen (jouer / to play)}
\begin{tabular}{ll c l l}
\toprule
\textbf{Pronom} & \textbf{Lëtzebuergesch} & \textbf{} & \textbf{Français} & \textbf{English} \\
\midrule
ech & spillen & $\rightarrow$ & je joue & I play \\
du & spills & $\rightarrow$ & tu joues & you play \\
hien/hatt/et & spillt & $\rightarrow$ & il/elle joue & he/she/it plays \\
mir & spillen & $\rightarrow$ & nous jouons & we play \\
dir & spillt & $\rightarrow$ & vous jouez & you play \\
si & spillen & $\rightarrow$ & ils/elles jouent & they play \\
\bottomrule
\end{tabular}

\subsection*{Spullen (rincer / to rinse)}
\begin{tabular}{ll c l l}
\toprule
\textbf{Pronom} & \textbf{Lëtzebuergesch} & \textbf{} & \textbf{Français} & \textbf{English} \\
\midrule
ech & spullen & $\rightarrow$ & je rince & I rinse \\
du & spulls & $\rightarrow$ & tu rinces & you rinse \\
hien/hatt/et & spullt & $\rightarrow$ & il/elle rince & he/she/it rinses \\
mir & spullen & $\rightarrow$ & nous rinçons & we rinse \\
dir & spullt & $\rightarrow$ & vous rincez & you rinse \\
si & spullen & $\rightarrow$ & ils/elles rincent & they rinse \\
\bottomrule
\end{tabular}

\subsection*{Stoen (se tenir debout / to stand)}
\begin{tabular}{ll c l l}
\toprule
\textbf{Pronom} & \textbf{Lëtzebuergesch} & \textbf{} & \textbf{Français} & \textbf{English} \\
\midrule
ech & stinn & $\rightarrow$ & je me tiens debout & I stand \\
du & stees & $\rightarrow$ & tu te tiens debout & you stand \\
hien/hatt/et & steet & $\rightarrow$ & il/elle se tient debout & he/she/it stands \\
mir & stinn & $\rightarrow$ & nous nous tenons debout & we stand \\
dir & stitt & $\rightarrow$ & vous vous tenez debout & you stand \\
si & stinn & $\rightarrow$ & ils/elles se tiennent debout & they stand \\
\bottomrule
\end{tabular}

\subsection*{Schneiden (couper / to cut)}
\begin{tabular}{ll c l l}
\toprule
\textbf{Pronom} & \textbf{Lëtzebuergesch} & \textbf{} & \textbf{Français} & \textbf{English} \\
\midrule
ech & schneiden & $\rightarrow$ & je coupe & I cut \\
du & schneids & $\rightarrow$ & tu coupes & you cut \\
hien/hatt/et & schneit & $\rightarrow$ & il/elle coupe & he/she/it cuts \\
mir & schneiden & $\rightarrow$ & nous coupons & we cut \\
dir & schneit & $\rightarrow$ & vous coupez & you cut \\
si & schneiden & $\rightarrow$ & ils/elles coupent & they cut \\
\bottomrule
\end{tabular}

\subsection*{Zerwéieren (servir / to serve)}
\begin{tabular}{ll c l l}
\toprule
\textbf{Pronom} & \textbf{Lëtzebuergesch} & \textbf{} & \textbf{Français} & \textbf{English} \\
\midrule
ech & zerwéieren & $\rightarrow$ & je sers & I serve \\
du & zerwéiers & $\rightarrow$ & tu sers & you serve \\
hien/hatt/et & zerwéiert & $\rightarrow$ & il/elle sert & he/she/it serves \\
mir & zerwéieren & $\rightarrow$ & nous servons & we serve \\
dir & zerwéiert & $\rightarrow$ & vous servez & you serve \\
si & zerwéieren & $\rightarrow$ & ils/elles servent & they serve \\
\bottomrule
\end{tabular}

\subsection*{Klammen (grimper / to climb)}
\begin{tabular}{ll c l l}
\toprule
\textbf{Pronom} & \textbf{Lëtzebuergesch} & \textbf{} & \textbf{Français} & \textbf{English} \\
\midrule
ech & klammen & $\rightarrow$ & je grimpe & I climb \\
du & klëms & $\rightarrow$ & tu grimpes & you climb \\
hien/hatt/et & klëmmt & $\rightarrow$ & il/elle grimpe & he/she/it climbs \\
mir & klammen & $\rightarrow$ & nous grimpons & we climb \\
dir & klammt & $\rightarrow$ & vous grimpez & you climb \\
si & klammen & $\rightarrow$ & ils/elles grimpent & they climb \\
\bottomrule
\end{tabular}

\subsection*{Halen (tenir / to hold)}
\begin{tabular}{ll c l l}
\toprule
\textbf{Pronom} & \textbf{Lëtzebuergesch} & \textbf{} & \textbf{Français} & \textbf{English} \\
\midrule
ech & halen & $\rightarrow$ & je tiens & I hold \\
du & häls & $\rightarrow$ & tu tiens & you hold \\
hien/hatt/et & hält & $\rightarrow$ & il/elle tient & he/she/it holds \\
mir & halen & $\rightarrow$ & nous tenons & we hold \\
dir & haalt & $\rightarrow$ & vous tenez & you hold \\
si & halen & $\rightarrow$ & ils/elles tiennent & they hold \\
\bottomrule
\end{tabular}


\section{Derivatives of 'klammen'}
\textbf{Erausklammen} \textit{sortir en grimpant / to climb out}\newline D'Kanner klammen aus dem Waasser eraus. \textit{(Les enfants grimpent hors de l'eau / The children climb out of the water.)}

\textbf{Eraklammen} \textit{entrer en grimpant / to climb in}\newline De Mann klëmmt an den Auto eraan. \textit{(L'homme monte dans la voiture / The man climbs into the car.)}

\textbf{Eropklammen} \textit{monter en grimpant / to climb up}\newline D'Kaz klëmmt op de Bam erop. \textit{(Le chat grimpe à l'arbre / The cat climbs up the tree.)}

\textbf{Erofklammen} \textit{descendre en grimpant / to climb down}\newline Si klëmmt vum Daach erof. \textit{(Elle descend du toit / She climbs down from the roof.)}

\textbf{Ausklammen} \textit{sortir en grimpant / to climb out of}\newline Hien klëmmt aus der Fënster eraus. \textit{(Il sort par la fenêtre en grimpant / He climbs out of the window.)}

\section{Dictionary}
\begin{multicols}{2}

\begin{minipage}[t]{0.48\textwidth}
    \centering
    \textbf{e Bouf}
    \vspace{0.5em}
    \framebox{\parbox{0.9\linewidth}{\centering
            \vspace{0.5cm}
            \includegraphics[width=0.9\linewidth]{vokab_300dpi/bouf.png} \\
            \small\textit{(garçon / boy)}
            \vspace{0.5cm}
    }}
\end{minipage}

\begin{minipage}[t]{0.48\textwidth}
    \centering
    \textbf{Bouwen}
    \vspace{0.5em}
    \framebox{\parbox{0.9\linewidth}{\centering
            \vspace{0.5cm}
            \includegraphics[width=0.9\linewidth]{vokab_300dpi/bouwen.png} \\
            \small\textit{(garçons / boys)}
            \vspace{0.5cm}
    }}
\end{minipage}

\begin{minipage}[t]{0.48\textwidth}
    \centering
    \textbf{eng Gaass}
    \vspace{0.5em}
    \framebox{\parbox{0.9\linewidth}{\centering
            \vspace{0.5cm}
            \includegraphics[width=0.9\linewidth]{vokab_300dpi/gaass.png} \\
            \small\textit{(ruelle / alley)}
            \vspace{0.5cm}
    }}
\end{minipage}

\begin{minipage}[t]{0.48\textwidth}
    \centering
    \textbf{e Wee}
    \vspace{0.5em}
    \framebox{\parbox{0.9\linewidth}{\centering
            \vspace{0.5cm}
            \includegraphics[width=0.9\linewidth]{vokab_300dpi/wee.png} \\
            \small\textit{(chemin / way, path)}
            \vspace{0.5cm}
    }}
\end{minipage}

\begin{minipage}[t]{0.48\textwidth}
    \centering
    \textbf{eng Luucht}
    \vspace{0.5em}
    \framebox{\parbox{0.9\linewidth}{\centering
            \vspace{0.5cm}
            \includegraphics[width=0.9\linewidth]{vokab_300dpi/luucht.png} \\
            \small\textit{(lumière / light)}
            \vspace{0.5cm}
    }}
\end{minipage}

\begin{minipage}[t]{0.48\textwidth}
    \centering
    \textbf{en Haus}
    \vspace{0.5em}
    \framebox{\parbox{0.9\linewidth}{\centering
            \vspace{0.5cm}
            \includegraphics[width=0.9\linewidth]{vokab_300dpi/haus.png} \\
            \small\textit{(maison / house)}
            \vspace{0.5cm}
    }}
\end{minipage}

\begin{minipage}[t]{0.48\textwidth}
    \centering
    \textbf{e Metzler}
    \vspace{0.5em}
    \framebox{\parbox{0.9\linewidth}{\centering
            \vspace{0.5cm}
            \includegraphics[width=0.9\linewidth]{vokab_300dpi/metzler.png} \\
            \small\textit{(boucher / butcher)}
            \vspace{0.5cm}
    }}
\end{minipage}

\begin{minipage}[t]{0.48\textwidth}
    \centering
    \textbf{e Jelli}
    \vspace{0.5em}
    \framebox{\parbox{0.9\linewidth}{\centering
            \vspace{0.5cm}
            \includegraphics[width=0.9\linewidth]{vokab_300dpi/jelli.png} \\
            \small\textit{(gelée / jelly)}
            \vspace{0.5cm}
    }}
\end{minipage}

\begin{minipage}[t]{0.48\textwidth}
    \centering
    \textbf{Stécker}
    \vspace{0.5em}
    \framebox{\parbox{0.9\linewidth}{\centering
            \vspace{0.5cm}
            \includegraphics[width=0.9\linewidth]{vokab_300dpi/stecker.png} \\
            \small\textit{(morceaux / pieces)}
            \vspace{0.5cm}
    }}
\end{minipage}

\begin{minipage}[t]{0.48\textwidth}
    \centering
    \textbf{e Stéck}
    \vspace{0.5em}
    \framebox{\parbox{0.9\linewidth}{\centering
            \vspace{0.5cm}
            \includegraphics[width=0.9\linewidth]{vokab_300dpi/steck.png} \\
            \small\textit{(morceau / piece)}
            \vspace{0.5cm}
    }}
\end{minipage}

\begin{minipage}[t]{0.48\textwidth}
    \centering
    \textbf{eng Bitchen}
    \vspace{0.5em}
    \framebox{\parbox{0.9\linewidth}{\centering
            \vspace{0.5cm}
            \includegraphics[width=0.9\linewidth]{vokab_300dpi/bitchen.png} \\
            \small\textit{(tonneau, saloir / barrel, salting tub)}
            \vspace{0.5cm}
    }}
\end{minipage}

\begin{minipage}[t]{0.48\textwidth}
    \centering
    \textbf{eng Dier}
    \vspace{0.5em}
    \framebox{\parbox{0.9\linewidth}{\centering
            \vspace{0.5cm}
            \includegraphics[width=0.9\linewidth]{vokab_300dpi/dier.png} \\
            \small\textit{(porte / door)}
            \vspace{0.5cm}
    }}
\end{minipage}

\begin{minipage}[t]{0.48\textwidth}
    \centering
    \textbf{eng Schëssel}
    \vspace{0.5em}
    \framebox{\parbox{0.9\linewidth}{\centering
            \vspace{0.5cm}
            \includegraphics[width=0.9\linewidth]{vokab_300dpi/schessel.png} \\
            \small\textit{(bol / bowl)}
            \vspace{0.5cm}
    }}
\end{minipage}

\begin{minipage}[t]{0.48\textwidth}
    \centering
    \textbf{Fleesch}
    \vspace{0.5em}
    \framebox{\parbox{0.9\linewidth}{\centering
            \vspace{0.5cm}
            \includegraphics[width=0.9\linewidth]{vokab_300dpi/fleesch.png} \\
            \small\textit{(viande / meat)}
            \vspace{0.5cm}
    }}
\end{minipage}
\end{multicols}