\chapter{Leçon 16}

\section{D'Deeg vun der Woch (Days of the Week / Les jours de la semaine)}

Hei sinn d'Deeg vun der Woch. Bedenkt datt et dacks verschidde Schreifweisen an Nimm gëtt (mat -deg oder -den). \\
Here are the days of the week. Note that there are often different spellings (with -deg or -den). \\
Voici les jours de la semaine. Notez qu'il y a souvent différentes orthographes (avec -deg ou -den).

\begin{itemize}
    \item \textbf{Méindeg / Méinden} (Monday / Lundi)
    \item \textbf{Dënschdeg / Dënschden} (Tuesday / Mardi)
    \item \textbf{Mëttwoch} (Wednesday / Mercredi)
    \item \textbf{Donneschdeg / Donneschden} (Thursday / Jeudi)
    \item \textbf{Freideg / Freiden} (Friday / Vendredi)
    \item \textbf{Samsdeg / Samschdeg / Samsden / Samschden} (Saturday / Samedi)
    \item \textbf{Sonndeg / Sonnden} (Sunday / Dimanche)
\end{itemize}

\subsection*{Zäitlinn (Timeline / Ligne du temps)}

\begin{center}
\begin{tikzpicture}[every node/.style={align=center, font=\sffamily}]
    % Draw timeline
    \draw[thick, ->] (-6.5,0) -- (6.5,0);
    
    % Draw ticks and nodes explicitly to avoid macro expansion issues in TikZ foreach
    % virugëschter (-5)
    \draw[thick] (-5,0.2) -- (-5,-0.2);
    \node[above] at (-5,0.3) {\small\bfseries virugëschter\\(Donneschdeg)};
    \node[below, text=gray] at (-5,-0.3) {\scriptsize day before yesterday\\avant-hier};

    % gëschter (-2.5)
    \draw[thick] (-2.5,0.2) -- (-2.5,-0.2);
    \node[above] at (-2.5,0.3) {\small\bfseries gëschter\\(Freideg)};
    \node[below, text=gray] at (-2.5,-0.3) {\scriptsize yesterday\\hier};

    % haut (0)
    \draw[thick] (0,0.2) -- (0,-0.2);
    \node[above] at (0,0.3) {\small\bfseries haut\\(Samsdeg)};
    \node[below, text=gray] at (0,-0.3) {\scriptsize today\\aujourd'hui};

    % muer (2.5)
    \draw[thick] (2.5,0.2) -- (2.5,-0.2);
    \node[above] at (2.5,0.3) {\small\bfseries muer\\(Sonndeg)};
    \node[below, text=gray] at (2.5,-0.3) {\scriptsize tomorrow\\demain};

    % iwwermuer (5)
    \draw[thick] (5,0.2) -- (5,-0.2);
    \node[above] at (5,0.3) {\small\bfseries iwwermuer\\(Méindeg)};
    \node[below, text=gray] at (5,-0.3) {\scriptsize day after tomorrow\\après-demain};
    
    % Example pointers
    \node[above=1.5cm, text=geminiBlue, font=\bfseries] (qx) at (0,0) {Wat ass haut fir en Dag?};
    \draw[->, geminiBlue, thick] (qx) -- (0,0.8);

    \node[below=1.6cm, text=purple, font=\bfseries] (qy) at (2.5,0) {Wat ass muer?};
    \draw[->, purple, thick] (qy) -- (2.5,-1.2);

    \node[below=1.6cm, text=magenta, font=\bfseries] (qz) at (-2.5,0) {Wat war gëschter?};
    \draw[->, magenta, thick] (qz) -- (-2.5,-1.2);
\end{tikzpicture}
\end{center}

\begin{itemize}
    \item \textbf{Wat ass haut fir en Dag? / Wat fir en Dag ass haut?} \\
    (What day is it today? / Quel jour sommes-nous aujourd'hui ?) \\
    \textit{Haut ass Samschdeg. (Today is Saturday. / Aujourd'hui c'est samedi.)}
    \item \textbf{Wat ass muer?} \\
    (What is tomorrow? / C'est quoi demain ?) \\
    \textit{Muer ass Sonndeg. (Tomorrow is Sunday. / Demain c'est dimanche.)}
    \item \textbf{Wat war gëschter?} \\
    (What was yesterday? / C'était quoi hier ?) \\
    \textit{Gëschter war Freideg. (Yesterday was Friday. / Hier c'était vendredi.)}
    \item \textbf{Wat war virugëschter?} \\
    (What was the day before yesterday? / C'était quoi avant-hier ?) \\
    \textit{Virugëschter war Donneschdeg. (The day before yesterday was Thursday. / Avant-hier c'était jeudi.)}
    \item \textbf{Wat ass iwwermuer?} \\
    (What is the day after tomorrow? / C'est quoi après-demain ?) \\
    \textit{Iwwermuer ass Méindeg. (The day after tomorrow is Monday. / Après-demain ce sera lundi.)}
\end{itemize}

\section{Froewierder (Question words / Mots interrogatifs)}

An dëser Lektioun weise mir wéi ee verschidde Froewierder benotzt. \\
In this lesson, we show how to use several question words. \\
Dans cette leçon, nous montrons comment utiliser plusieurs mots interrogatifs.

\subsection{Wat (What / Que, Quoi)}

"Wat" gëtt benotzt fir no Saachen oder Aktivitéiten ze froen. \\
"Wat" is used to ask about things or activities. \\
« Wat » est utilisé pour poser des questions sur des choses ou des activités.

\begin{itemize}
    \item \textbf{Wat kaacht den Alex?} \\
    (What is Alex cooking? / Que cuisine Alex ?) \\
    \textit{Hie kaacht Poulet. (He is cooking chicken. / Il cuisine du poulet.)}
    
    \item \textbf{Wat méchs du?} \\
    (What are you doing? / Que fais-tu ?) \\
    \textit{Ech schaffen hei. (I work here. / Je travaille ici.)}
    
    \item \textbf{Wat ass (hei) lass?} \\
    (What's going on (here)? / Que se passe-t-il (ici) ?) 
    
    \item \textbf{Wat?} \\
    (What? / Quoi ?) \textit{Opgepasst: Dat ass net ganz héiflech! / Warning: This is not very polite! / Attention : Ce n'est pas très poli !} \\
    \textit{Besser (Better / Mieux): Wéi wannechgelift? (Pardon? / Pardon ?)}
    
    \item \textbf{Wat wëlls du drénken?} \\
    (What do you want to drink? / Que veux-tu boire ?) \\
    \textit{Ech huelen e Waasser, wgl. (I'll have a water, please. / Je prends une eau, s.v.p.)}
\end{itemize}

\subsection{Wat fir en / eng... oder Wéi en / eng... (Which, What kind of / Quel, Quelle)}

Dës Ausdréck benotzt ee fir eng Choix ze froen oder Detailer iwwer eppes ze kréien. \\
These expressions are used to ask for a choice or details about something. \\
Ces expressions sont utilisées pour demander un choix ou des détails sur quelque chose.

\begin{itemize}
    \item \textbf{Wéi eng Faarf?} \\
    (Which color? / Quelle couleur ?)
    
    \item \textbf{Wat ass haut fir en Dag? / Wat fir en Dag ass haut?} \\
    (What day is it today? / Quel jour sommes-nous aujourd'hui ?) \\
    \textit{Haut ass Samschdeg. (Today is Saturday. / Aujourd'hui c'est samedi.)}
    
    \item \textbf{Wéi en Auto hues du?} \\
    (Which car do you have? / Quelle voiture as-tu ?)
    
    \item \textbf{Wat fir e Buch lies du grad?} \\
    (What kind of book are you reading right now? / Quel genre de livre lis-tu en ce moment ?)
    
    \item \textbf{Wéi eng Gréisst hutt Dir?} \\
    (What size do you have (wear)? / Quelle taille faites-vous ?) \\
    \textit{Ech hu Gréisst 42. (I am size 42. / Je fais du 42.)}
\end{itemize}

\subsection{Wou (Where / Où)}

"Wou" gëtt benotzt fir no enger Plaz oder enger Positioun ze froen. \\
"Wou" is used to ask about a place or position. \\
« Wou » est utilisé pour poser des questions sur un lieu ou une position.

\begin{itemize}
    \item \textbf{Wou wunns du?} \\
    (Where do you live? / Où habites-tu ?) \\
    \textit{Ech wunnen an der Stad. (I live in the city. / J'habite en ville.)}
    
    \item \textbf{Wou ass d'Buch?} \\
    (Where is the book? / Où est le livre ?) \\
    \textit{Et läit op dem Dësch. (It is lying on the table. / Il est posé sur la table.)}
    
    \item \textbf{Wou ass mäi Brëll?} \\
    (Where are my glasses? / Où sont mes lunettes ?) \\
    \textit{Op dem Kapp! (On your head! / Sur ta tête !)}
    
    \item \textbf{Wou ass d'Gare, wgl.?} \\
    (Where is the train station, please? / Où est la gare, s.v.p. ?)
    
    \item \textbf{Wou schaffs du?} \\
    (Where do you work? / Où travailles-tu ?)
\end{itemize}

\subsection{Vu wou (From where / D'où)}

"Vu wou" gëtt benotzt fir no der Hierkonft (Origin) ze froen. \\
"Vu wou" is used to ask about the origin. \\
« Vu wou » est utilisé pour poser des questions sur l'origine.

\begin{itemize}
    \item \textbf{Vu wou kënns du?} \\
    (Where are you from? / D'où viens-tu ?) \\
    \textit{Ech komme vu Frankräich. (I come from France. / Je viens de France.)}
    
    \item \textbf{Vu wou rufft Dir un?} \\
    (Where are you calling from? / D'où appelez-vous ?)
    
    \item \textbf{Vu wou hues du dat?} \\
    (Where did you get that from? / D'où as-tu ça ?)
\end{itemize}

\subsection{Wouhin / Wuer (Where to / Vers où, Où (direction))}

Dës Wierder froen no enger Richtung oder enger Destinatioun. \\
These words ask about a direction or destination. \\
Ces mots posent des questions sur une direction ou une destination.

\begin{itemize}
    \item \textbf{Wou gi mir hin?} \\
    (Where are we going? / Où allons-nous ?)
    
    \item \textbf{Wouhi muss ee goe fir e Bréif ze schécken?} \\
    (Where must one go to send a letter? / Où doit-on aller pour envoyer une lettre ?) \\
    \textit{Op d'Post. (To the post office. / À la poste.)}
    
    \item \textbf{Wouhi fiers du?} \\
    (Where are you driving to? / Où vas-tu (en conduisant) ?) \\
    \textit{Ech fueren heem. (I am driving home. / Je rentre à la maison.)} \\
    \textit{Ech fueren och heem, ech komme mat. (I am also driving home, I am coming with you. / Je rentre aussi, je viens avec toi.)}
    
    \item \textbf{Wouhi geet dee Wee?} \\
    (Where does this path go? / Où mène ce chemin ?)
    
    \item \textbf{Wuer geet den Trapp?} \\
    (Where do the stairs go? / Où mènent les escaliers ?)
\end{itemize}

\subsection{Wéi (How / Comment)}

"Wéi" freet no der Aart a Weis oder engem Zoustand. \\
"Wéi" asks about the manner or a condition. \\
« Wéi » pose des questions sur la manière ou une condition.

\begin{itemize}
    \item \textbf{Wéi geet et?} \\
    (How are you? / Comment ça va ?)
    
    \item \textbf{Wéi gëtt d'Wieder muer?} \\
    (How will the weather be tomorrow? / Quel temps fera-t-il demain ?)
    
    \item \textbf{Wéi heeschst du?} \\
    (What is your name? [lit: How are you called?] / Comment t'appelles-tu ?)
    
    \item \textbf{Wéi laang dauert dat?} \\
    (How long does that take? / Combien de temps ça prend ?)
    
    \item \textbf{Wéi funktionéiert dat?} \\
    (How does that work? / Comment ça fonctionne ?)
\end{itemize}

\section{E puer Wierder an Ausdréck (Some words and phrases / Quelques mots et expressions)}

\begin{itemize}
    \item \textbf{Ech hunn dat léiwer.} \\
    (I prefer that. / Je préfère ça.)
    
    \item \textbf{Ech kache léiwer doheem.} \\
    (I prefer cooking at home. / Je préfère cuisiner à la maison.)
    
    \item \textbf{Ech drénke Wäin, mee ech hu léiwer Béier.} \\
    (I drink wine, but I prefer beer. / Je bois du vin, mais je préfère la bière.)
    
    \item \textbf{eng Drëpp} (eng Drëpp drénken) $\rightarrow$ \textit{Plural: Drëppen} \\
    (A shot of liquor [Schnapps] / Une petite goutte [verre de goutte/schnaps])
    
    \item \textbf{eng Drëps} $\rightarrow$ \textit{Plural: Drëpsen} \\
    (A drop [of liquid like water/rain] / Une goutte [d'un liquide comme l'eau/pluie])
    
    \item \textbf{Wéini gëtt et z'iessen?} \\
    (When will there be food? / Quand est-ce qu'on mange ?) \\
    \textit{Wann et fäerdeg ass. (When it's ready. / Quand c'est prêt.)} \\
    \textit{An der Mëttesstonn. (At lunchtime. / À l'heure de midi.)}
    
    \item \textbf{Wéini ass den nächste Lëtzebuergesche Cours?} \\
    (When is the next Luxembourgish course? / Quand est le prochain cours de luxembourgeois ?)
    
    \item \textbf{Firwat bass du net komm?} \\
    (Why didn't you come? / Pourquoi n'es-tu pas venu ?)
    
    \item \textbf{Ech hunn dat net gesot.} \\
    (I didn't say that. / Je n'ai pas dit ça.)
    
    \item \textbf{Turnschlappen = Turnschong} \\
    (Sneakers, gym shoes / Baskets, chaussures de sport)
\end{itemize}

\section{Gell?}

Den Ausdrock "Gell?" gëtt ganz dacks um Enn vun engem Saz benotzt, fir eng Zoustëmmung (Bestätegung) ze froen. Et entsprécht dem "N'est-ce pas ?" op Franséisch oder "Right? / Isn't it?" op Englesch. \\
The expression "Gell?" is very often used at the end of a sentence to ask for agreement or confirmation. \\
L'expression « Gell ? » est très souvent utilisée à la fin d'une phrase pour demander une confirmation.

Ofhängeg dovun, mat wiem Dir schwätzt, gëtt dëst Wuert ugepasst (konjugéiert): \\
Depending on who you are talking to, this word adapts (is conjugated): \\
Selon la personne à qui vous parlez, ce mot s'adapte (se conjugue) :

\begin{itemize}
    \item \textbf{Gell? / Në?} \\
    (General form / Forme générale)
    
    \item \textbf{Gelldu? / Gedu? / Gelu? / Nedu?} \\
    \textit{Et benotzt ee mat "du". (Familiar form / Forme familière)} \\
    Beispill: Du kënns muer och op d'Party, \textbf{gelldu}? \\
    (You are also coming to the party tomorrow, right? / Tu viens aussi à la fête demain, n'est-ce pas ?)
    
    \item \textbf{Gellt? / Gediert? / Gelldiert? / Nediert?} \\
    \textit{Et benotzt ee mat "Dir". (Formal/Plural form / Forme formelle/pluriel)} \\
    Beispill: Dir hutt d'Hausaufgaben gemaach, \textbf{gellt}? \\
    (You did the homework, right? / Vous avez fait les devoirs, n'est-ce pas ?)
\end{itemize}
