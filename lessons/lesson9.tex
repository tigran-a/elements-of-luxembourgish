\chapter{Leçon 9}

\section{Verben / Verbs / Verbes}

\subsection{Sinn (être / to be)}\begin{tabular}{ll c l l}
\toprule
\textbf{Pronom} & \textbf{Lëtzebuergesch} & \textbf{} & \textbf{Français} & \textbf{English} \\
\midrule
 ech & sinn & $\rightarrow$ & je suis & I am \\
 du & bass & $\rightarrow$ & tu es & you are \\
 hien/hatt/si & ass & $\rightarrow$ & il/elle est & he/she/it is \\
 mir & sinn & $\rightarrow$ & nous sommes & we are \\
 dir & sidd & $\rightarrow$ & vous êtes & you are \\
 si & sinn & $\rightarrow$ & ils/elles sont & they are \\
\bottomrule
\end{tabular}

\subsection{Hunn (avoir / to have)}\begin{tabular}{ll c l l}
\toprule
\textbf{Pronom} & \textbf{Lëtzebuergesch} & \textbf{} & \textbf{Français} & \textbf{English} \\
\midrule
 ech & hunn & $\rightarrow$ & j'ai & I have \\
 du & hues & $\rightarrow$ & tu as & you have \\
 hien/hatt/si & huet & $\rightarrow$ & il/elle a & he/she/it has \\
 mir & hunn & $\rightarrow$ & nous avons & we have \\
 dir & hutt & $\rightarrow$ & vous avez & you have \\
 si & hunn & $\rightarrow$ & ils/elles ont & they have \\
\bottomrule
\end{tabular}

\subsection{Wëllen (vouloir / to want)}\begin{tabular}{ll c l l}
\toprule
\textbf{Pronom} & \textbf{Lëtzebuergesch} & \textbf{} & \textbf{Français} & \textbf{English} \\
\midrule
 ech & wëll & $\rightarrow$ & je veux & I want \\
 du & wëlls & $\rightarrow$ & tu veux & you want \\
 hien/hatt/si & wëll & $\rightarrow$ & il/elle veut & he/she/it wants \\
 mir & wëllen & $\rightarrow$ & nous voulons & we want \\
 dir & wëllt & $\rightarrow$ & vous voulez & you want \\
 si & wëllen & $\rightarrow$ & ils/elles veulent & they want \\
\bottomrule
\end{tabular}

\section{Virstellen / Introduction / Présentation}

\begin{itemize}
    \item \textbf{Ech sinn d'Eugénie.} \\
    \textit{(Je suis Eugénie. / I am Eugénie.)}
    
    \item \textbf{Ech sinn 20 (zwanzeg) Joer al.} \\
    \textit{(J'ai 20 ans. / I am 20 years old.)}
    
    \item \textbf{Ech si Franséisin.} \\
    \textit{(Je suis française. / I am French.)}
    
    \item \textbf{Ech sinn den Alex.} \\
    \textit{(Je suis Alex. / I am Alex.)}
    
    \item \textbf{Ech sinn 30 (drësseg) Joer al.} \\
    \textit{(J'ai 30 ans. / I am 30 years old.)}
    
    \item \textbf{Ech sinn Togolees.} \\
    \textit{(Je suis togolais. / I am Togolese.)}
    
    \item \textbf{Vu wou kënns du?} \\
    \textit{(D'où viens-tu ? / Where are you from?)}
\end{itemize}

\subsection{Nationalitéiten / Nationalités / Nationalities}
\begin{itemize}
    \item \textbf{de Fransous / d'Franséisin} \textit{(le Français / la Française - the Frenchman / the Frenchwoman)}
    \item \textbf{den Togolees / d'Togoleesin} \textit{(le Togolais / la Togolaise - the Togolese man / the Togolese woman)}
\end{itemize}

\subsection{Zoustand / État / State}
\begin{itemize}
    \item \textbf{Ech si prett.} \\
    \textit{(Je suis prêt(e). / I am ready.)}
    
    \item \textbf{Bass du midd?} \\
    \textit{(Es-tu fatigué(e) ? / Are you tired?)}
\end{itemize}

\section{Verneinung / Négation / Negation}

In Luxembourgish, there are two main ways to say "not" or "no": \textbf{net} and \textbf{keen/keng}.
\textit{(En luxembourgeois, il y a deux façons principales de dire "ne... pas" ou "pas de" : \textbf{net} et \textbf{keen/keng}.)}

\subsection{Net (ne... pas / not)}
Used with verbs, adjectives, and proper nouns.
\textit{(Utilisé avec les verbes, les adjectifs et les noms propres.)}

\begin{itemize}
    \item \textbf{Ech si net prett.} \\
    \textit{(Je ne suis pas prêt. / I am not ready.)}
    
    \item \textbf{Ech sinn net den Alex.} \\
    \textit{(Je ne suis pas Alex. / I am not Alex.)}
\end{itemize}

\subsection{Keen / Keng (pas de / no)}
Used to negate nouns (like indefinite articles "en/eng").
\textit{(Utilisé pour nier les noms, remplace les articles indéfinis "un/une".)}

\begin{itemize}
    \item \textbf{Ech hunn en Auto.} $\rightarrow$ \textbf{Ech hu keen Auto.} \\
    \textit{(J'ai une voiture. $\rightarrow$ Je n'ai pas de voiture. / I have a car. $\rightarrow$ I have no car.)}
\end{itemize}

\section{Wierder a Sätz / Mots et Phrases / Words and Sentences}

\subsection{Suen / Argent / Money}
\begin{itemize}
    \item \textbf{Ech hu Suen / Geld.} \\
    \textit{(J'ai de l'argent. / I have money.)}
    
    \item \textbf{Ech wëll vill Suen hunn.} \\
    \textit{(Je veux avoir beaucoup d'argent. / I want to have a lot of money.)}
    
    \item \textbf{Ech bezuelen.} \\
    \textit{(Je paie. / I pay.)}
    
    \item \textbf{Et ass genuch.} \\
    \textit{(C'est assez. / It is enough.)}
\end{itemize}

\subsection{Spillen a Wanter / Jouer et Hiver / Playing and Winter}
\begin{itemize}
    \item \textbf{Fussball / Futtball / Foussball} \\
    \textit{(Football / Football / Soccer)}
    
    \item \textbf{e Ball / Bäll} \\
    \textit{(un ballon / des balles - a ball / balls)}
    
    \item \textbf{e Puzzle / e Puzzlestéck / Puzzlestécker} \\
    \textit{(un puzzle / une pièce de puzzle / des pièces de puzzle - a puzzle / a puzzle piece / puzzle pieces)}
    
    \item \textbf{Ech hunn 1000 (dausend) Puzzlestécker.} \\
    \textit{(J'ai 1000 pièces de puzzle. / I have 1000 puzzle pieces.)}
    
    \item \textbf{Et ass Wanter.} \\
    \textit{(C'est l'hiver. / It is winter.)}
    
    \item \textbf{Et ass kal.} \\
    \textit{(Il fait froid. / It is cold.)}
\end{itemize}

\subsection{Zuelen a Quantitéiten / Nombres et Quantités / Numbers and Quantities}
\begin{itemize}
    \item \textbf{E Ball, zwee Bäll.} \\
    \textit{(Une balle, deux balles. / One ball, two balls.)}
    
    \item \textbf{En Auto, zwee Autoen, vill Autoen.} \\
    \textit{(Une voiture, deux voitures, beaucoup de voitures. / One car, two cars, many cars.)}
\end{itemize}

\subsection{Kuerz Wierder / Mots Courts / Short Words}
\begin{itemize}
    \item \textbf{Jo} \textit{(Oui / Yes)}
    \item \textbf{Nee} \textit{(Non / No)}
    \item \textbf{Mee / Mä / Awer} \textit{(Mais / But)}
\end{itemize}

\section{Quiz}

\subsection{Froen (English / Français)}
Translate the following sentences into Luxembourgish. / Traduisez les phrases suivantes en luxembourgeois.

\begin{enumerate}
    \item Are you ready? I am ready, but you are not ready. \\ \textit{(Es-tu prêt ? Je suis prêt, mais tu n'es pas prêt.)}
    \item Does he have a ball? Yes, he has a ball. \\ \textit{(A-t-il une balle ? Oui, il a une balle.)}
    \item Does he have 27 balls? No, he doesn't have 27 balls. \\ \textit{(A-t-il 27 balles ? Non, il n'a pas 27 balles.)}
    \item I have a car, but I want to have 30 cars. \\ \textit{(J'ai une voiture, mais je veux avoir 30 voitures.)}
    \item 30 cars is enough, but 27 cars is not enough. \\ \textit{(30 voitures, c'est assez, mais 27 voitures, ce n'est pas assez.)}
    \item I want to have a lot of money. Do you have a lot of money? \\ \textit{(Je veux avoir beaucoup d'argent. As-tu beaucoup d'argent ?)}
    \item No, but I have 30000 puzzle pieces. \\ \textit{(Non, mais j'ai 30000 pièces de puzzle.)}
    \item I want to play football. Do you want to play football? \\ \textit{(Je veux jouer au football. Veux-tu jouer au football ?)}
    \item No, I have no ball. \\ \textit{(Non, je n'ai pas de balle.)}
    \item I want to make a puzzle. \\ \textit{(Je veux faire un puzzle.)}
    \item It is winter and it is cold. I want a car. \\ \textit{(C'est l'hiver et il fait froid. Je veux une voiture.)}
    \item I am not tired. I want to play. \\ \textit{(Je ne suis pas fatigué. Je veux jouer.)}
    \item He has 1000 balls. It is enough. \\ \textit{(Il a 1000 balles. C'est assez.)}
    \item Do you want to pay? No, I have no money. \\ \textit{(Veux-tu payer ? Non, je n'ai pas d'argent.)}
    \item She is 30 years old and she is French. \\ \textit{(Elle a 30 ans et elle est française.)}
    \item I am not Eugénie. I am Alex and I have money. \\ \textit{(Je ne suis pas Eugénie. Je suis Alex et j'ai de l'argent.)}
    \item We have a puzzle, but we have no puzzle pieces. \\ \textit{(Nous avons un puzzle, mais nous n'avons pas de pièces de puzzle.)}
    \item Are you Togolese? No, I am French. \\ \textit{(Es-tu togolais ? Non, je suis français.)}
    \item I want two cars. I have one car. \\ \textit{(Je veux deux voitures. J'ai une voiture.)}
    \item You are ready, but he is not ready. \\ \textit{(Tu es prêt, mais il n'est pas prêt.)}
    \item I want 7000 puzzle pieces. \\ \textit{(Je veux 7000 pièces de puzzle.)}
    \item He is 37 years old. \\ \textit{(Il a 37 ans.)}
    \item Do you have a car? No, I have 20 cars! \\ \textit{(As-tu une voiture ? Non, j'ai 20 voitures !)}
    \item It is cold, but I want to play football. \\ \textit{(Il fait froid, mais je veux jouer au football.)}
    \item We are ready. Are you ready? \\ \textit{(Nous sommes prêts. Êtes-vous prêts ?)}
    \item I have 27 euros. It is enough. \\ \textit{(J'ai 27 euros. C'est assez.)}
    \item She wants 1000 balls. \\ \textit{(Elle veut 1000 balles.)}
    \item Are you tired? No, I am not tired. \\ \textit{(Es-tu fatigué ? Non, je ne suis pas fatigué.)}
    \item I am 20 years old and I am Togolese. \\ \textit{(J'ai 20 ans et je suis togolais.)}
    \item Do you have money? Yes, I have 30000 euros. \\ \textit{(As-tu de l'argent ? Oui, j'ai 30000 euros.)}
\end{enumerate}

\subsection{Léisungen (Lëtzebuergesch)}

\begin{enumerate}
    \item \textbf{Bass du prett? Ech si prett, mee du bass net prett.} \\ \textit{(Are you ready? I am ready, but you are not ready.)}
    \item \textbf{Huet hien e Ball? Jo, hien huet e Ball.} \\ \textit{(Does he have a ball? Yes, he has a ball.)}
    \item \textbf{Huet hien 27 (siwenanzwanzeg) Bäll? Nee, hien huet keng 27 Bäll.} \\ \textit{(Does he have 27 balls? No, he doesn't have 27 balls.)}
    \item \textbf{Ech hunn en Auto, mee ech wëll 30 (drësseg) Autoen hunn.} \\ \textit{(I have a car, but I want to have 30 cars.)}
    \item \textbf{30 Autoen ass genuch, mee 27 Autoen ass net genuch.} \\ \textit{(30 cars is enough, but 27 cars is not enough.)}
    \item \textbf{Ech wëll vill Suen hunn. Hues du vill Suen?} \\ \textit{(I want to have a lot of money. Do you have a lot of money ?)}
    \item \textbf{Nee, mee ech hunn 30000 (drëssegdausend) Puzzlestécker.} \\ \textit{(No, but I have 30000 puzzle pieces.)}
    \item \textbf{Ech wëll Fussball spillen. Wëlls du Fussball spillen?} \\ \textit{(I want to play football. Do you want to play football?)}
    \item \textbf{Nee, ech hu kee Ball.} \\ \textit{(No, I have no ball.)}
    \item \textbf{Ech wëll e Puzzle maachen.} \\ \textit{(I want to make a puzzle.)}
    \item \textbf{Et ass Wanter an et ass kal. Ech wëll en Auto.} \\ \textit{(It is winter and it is cold. I want a car.)}
    \item \textbf{Ech sinn net midd. Ech wëll spillen.} \\ \textit{(I am not tired. I want to play.)}
    \item \textbf{Hien huet 1000 (dausend) Bäll. Et ass genuch.} \\ \textit{(He has 1000 balls. It is enough.)}
    \item \textbf{Wëlls du bezuelen? Nee, ech hu keng Suen (kee Geld).} \\ \textit{(Do you want to pay? No, I have no money.)}
    \item \textbf{Si ass 30 (drësseg) Joer al an si ass Franséisin.} \\ \textit{(She is 30 years old and she is French.)}
    \item \textbf{Ech sinn net d'Eugénie. Ech sinn den Alex an ech hu Suen.} \\ \textit{(I am not Eugénie. I am Alex and I have money.)}
    \item \textbf{Mir hunn e Puzzle, mee mir hu keng Puzzlestécker.} \\ \textit{(We have a puzzle, but we have no puzzle pieces.)}
    \item \textbf{Bass du Togolees? Nee, ech si Fransous.} \\ \textit{(Are you Togolese? No, I am French.)}
    \item \textbf{Ech wëll zwee Autoen. Ech hunn en Auto.} \\ \textit{(I want two cars. I have one car.)}
    \item \textbf{Du bass prett, mee hien ass net prett.} \\ \textit{(You are ready, but he is not ready.)}
    \item \textbf{Ech wëll 7000 (siwendeusend) Puzzlestécker hunn.} \\ \textit{(I want 7000 puzzle pieces.)}
    \item \textbf{Hien ass 37 (siwenandrësseg) Joer al.} \\ \textit{(He is 37 years old.)}
    \item \textbf{Hues du en Auto? Nee, ech hunn 20 (zwanzeg) Autoen!} \\ \textit{(Do you have a car? No, I have 20 cars !)}
    \item \textbf{Et ass kal, mee ech wëll Fussball spillen.} \\ \textit{(It is cold, but I want to play football.)}
    \item \textbf{Mir si prett. Sidd dir prett?} \\ \textit{(We are ready. Are you ready ?)}
    \item \textbf{Ech hunn 27 (siwenanzwanzeg) Euro. Et ass genuch.} \\ \textit{(I have 27 euros. It is enough.)}
    \item \textbf{Si wëll 1000 (dausend) Bäll hunn.} \\ \textit{(She wants 1000 balls.)}
    \item \textbf{Bass du midd? Nee, ech sinn net midd.} \\ \textit{(Are you tired? No, I am not tired.)}
    \item \textbf{Ech sinn 20 (zwanzeg) Joer al an ech sinn Togolees.} \\ \textit{(I am 20 years old and I am Togolese.)}
    \item \textbf{Hues du Suen? Jo, ech hunn 30000 (drëssegdausend) Euro.} \\ \textit{(Do you have money? Yes, I have 30000 euros.)}
\end{enumerate}

\section{Dictionary}
\begin{multicols}{2}

\begin{minipage}[t]{0.48\textwidth}
    \centering
    \textbf{Fussball / Foussball / Futtball }
    \vspace{0.5em}
    \framebox{\parbox{0.9\linewidth}{\centering
            \vspace{0.5cm}
            \includegraphics[width=0.9\linewidth]{vokab_300dpi/fussball.png} \\
            \small\textit{(football / soccer)}
            \vspace{0.5cm}
    }}
\end{minipage}

\begin{minipage}[t]{0.48\textwidth}
    \centering
    \textbf{e Ball}
    \vspace{0.5em}
    \framebox{\parbox{0.9\linewidth}{\centering
            \vspace{0.5cm}
            \includegraphics[width=0.9\linewidth]{vokab_300dpi/ball.png} \\
            \small\textit{(balle / ball)}
            \vspace{0.5cm}
    }}
\end{minipage}

\begin{minipage}[t]{0.48\textwidth}
    \centering
    \textbf{Bäll}
    \vspace{0.5em}
    \framebox{\parbox{0.9\linewidth}{\centering
            \vspace{0.5cm}
            \includegraphics[width=0.9\linewidth]{vokab_300dpi/baell.png} \\
            \small\textit{(balles / balls)}
            \vspace{0.5cm}
    }}
\end{minipage}

\begin{minipage}[t]{0.48\textwidth}
    \centering
    \textbf{e Puzzle}
    \vspace{0.5em}
    \framebox{\parbox{0.9\linewidth}{\centering
            \vspace{0.5cm}
            \includegraphics[width=0.9\linewidth]{vokab_300dpi/puzzle.png} \\
            \small\textit{(puzzle / puzzle)}
            \vspace{0.5cm}
    }}
\end{minipage}

\begin{minipage}[t]{0.48\textwidth}
    \centering
    \textbf{e Puzzlestéck}
    \vspace{0.5em}
    \framebox{\parbox{0.9\linewidth}{\centering
            \vspace{0.5cm}
            \includegraphics[width=0.9\linewidth]{vokab_300dpi/puzzlesteck.png} \\
            \small\textit{(pièce de puzzle / puzzle piece)}
            \vspace{0.5cm}
    }}
\end{minipage}

\begin{minipage}[t]{0.48\textwidth}
    \centering
    \textbf{Puzzlestécker}
    \vspace{0.5em}
    \framebox{\parbox{0.9\linewidth}{\centering
            \vspace{0.5cm}
            \includegraphics[width=0.9\linewidth]{vokab_300dpi/puzzlestecker.png} \\
            \small\textit{(pièces de puzzle / puzzle pieces)}
            \vspace{0.5cm}
    }}
\end{minipage}

\begin{minipage}[t]{0.48\textwidth}
    \centering
    \textbf{Wanter}
    \vspace{0.5em}
    \framebox{\parbox{0.9\linewidth}{\centering
            \vspace{0.5cm}
            \includegraphics[width=0.9\linewidth]{vokab_300dpi/wanter.png} \\
            \small\textit{(hiver / winter)}
            \vspace{0.5cm}
    }}
\end{minipage}

\end{multicols}