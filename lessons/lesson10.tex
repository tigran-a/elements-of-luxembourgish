\chapter{Leçon 10}

\section{ Ausdréck / Expressions  /  Expressions}

\begin{itemize}
    \item \textbf{Et ass net erlaabt.} \\
    \textit{(Ce n'est pas permis. / It is not allowed.)}
    
    \item \textbf{Wéi geet et dir? (Informal) / Wéi geet et Iech? (Formal)} \\
    \textit{(Comment vas-tu ? / Comment allez-vous ? / How are you?)}
    
    \item \textbf{Dat ass Alles.} \\
    \textit{(C'est tout. / That is all.)}
    
    \item \textbf{Alles Guddes am neie Joer.} \\
    \textit{(Meilleurs vœux pour la nouvelle année. / Best wishes for the New Year.)}
    
    \item \textbf{Ech verstinn näicht.} \\
    \textit{(Je ne comprends rien. / I understand nothing.)} \\
    \small{\textit{Note: When using "näicht" (nothing), you do not need the additional negation "net". / Note : Lorsque vous utilisez "näicht" (rien), vous n'avez pas besoin de la négation supplémentaire "net".}}
    
    \item \textbf{Ech schwätze nach net Lëtzebuergesch.} \\
    \textit{(Je ne parle pas encore luxembourgeois. / I don't speak Luxembourgish yet.)}
\end{itemize}

\section{Aussprooch: Laang a Kuerz Vokaler / Prononciation : Voyelles Longues et Courtes / Pronunciation: Long and Short Vowels}

In Luxembourgish, the spelling often tells you if a vowel is long or short based on what follows it.
\textit{(En luxembourgeois, l'orthographe indique souvent si une voyelle est longue ou courte en fonction de ce qui suit.)}

\subsection{D'Reegel / La Règle / The Rule}

\begin{enumerate}
    \item \textbf{Single Vowel + Single Consonant} $\rightarrow$ \textbf{Long Sound} \\
    If a single vowel is followed by only one consonant, the vowel is usually long. \\
    \textit{(Si une voyelle simple est suivie d'une seule consonne, la voyelle est généralement longue.)}
    
    \item \textbf{Double Consonant or Cluster} $\rightarrow$ \textbf{Short Vowel} \\
    If a vowel is followed by two or more consonants (like \textit{ll, ss, cht, bt}), it becomes short. \\
    \textit{(Si une voyelle est suivie de deux consonnes ou plus, elle devient courte.)}
    
    \item \textbf{Double Vowel + Cluster} $\rightarrow$ \textbf{Long Sound} \\
    To keep a vowel long before a consonant cluster, we must write it double. \\
    \textit{(Pour garder une voyelle longue devant un groupe de consonnes, nous devons l'écrire en double.)}
\end{enumerate}

\subsection{Beispiller / Exemples / Examples}

\begin{itemize}
    \item \textbf{al} vs. \textbf{all}
    \begin{itemize}
        \item \textbf{al} (vieux/old): 1 'a' + 1 'l' $\rightarrow$ \textbf{Long} 'a'.
        \item \textbf{all} (tout/all): 1 'a' + 2 'l's $\rightarrow$ \textbf{Short} 'a'.
    \end{itemize}
    
    \item \textbf{üben} vs. \textbf{geüübt}
    \begin{itemize}
        \item \textbf{üben} (pratiquer/to practice): 1 'ü' + 1 'b' $\rightarrow$ \textbf{Long} 'ü'.
        \item \textbf{geüübt} (pratiqué/practiced): The 'b' is now followed by 't' (cluster 'bt'). To keep the 'ü' \textbf{Long}, we must write it double: \textbf{üü}.
    \end{itemize}

    \item \textbf{weisen} vs. \textbf{wëssen}
    \begin{itemize}
        \item \textbf{weisen} (montrer/to show): The diphthong 'ei' is always long.
        \item \textbf{wëssen} (savoir/to know): 1 'ë' + 2 's's $\rightarrow$ \textbf{Short} 'ë'.
    \end{itemize}
\end{itemize}

\section{Verben / Verbes / Verbs}

\subsection{Ënnerscheeder / Distinctions / Differences}

\begin{description}
    \item[weisen vs. wëssen] \hfill \\
    \textbf{weisen}: montrer / to show \\
    \textit{Kanns du mir weisen? (Peux-tu me montrer ? / Can you show me?)} \\
    \textbf{wëssen}: savoir / to know \\
    \textit{Ech weess et net. (Je ne le sais pas. / I don't know it.)}
    
    \item[goen vs. ginn] \hfill \\
    \textbf{goen}: aller / to go \\
    \textit{Ech ginn heem. (Je vais à la maison. / I am going home.)} \\
    \textbf{ginn}: donner / to give \\
    \textit{Kanns du mir deng Gittar ginn? (Peux-tu me donner ta guitare ? / Can you give me your guitar?)}
\end{description}

\subsection{Aner Verben / Autres Verbes / Other Verbs}

\begin{itemize}
    \item \textbf{üüben / üben}: pratiquer / to practice \\
    \textit{Hues du geüübt? (As-tu pratiqué ? / Did you practice?)}
    
    \item \textbf{verstoen}: comprendre / to understand \\
    \textit{Firwat versteet hatt hien net? (Pourquoi ne le comprend-elle pas ? / Why does she not understand him?)}
    
    \item \textbf{bréngen}: apporter / to bring \\
    \textit{Ech sinn komm fir Keramik ze bréngen. (Je suis venu pour apporter de la céramique. / I came to bring ceramics.)}
    
    \item \textbf{iwwerweisen}: virer (argent) / to transfer (money)
    
    \item \textbf{stëmmen}: accorder (instrument) / to tune \\
    \textit{...fir ze stëmmen. (...pour accorder. / ...to tune.)}
\end{itemize}

\section{Grammaire / Grammaire / Grammar}

\subsection{Dat vs. Datt}
\begin{itemize}
    \item \textbf{dat} (pron. démonstratif / demonstrative pronoun): ce, ça / that \\
    \textit{Dat ass gutt. (C'est bien. / That is good.)}
    \item \textbf{datt} (conjonction / conjunction): que / that \\
    \textit{Ech mengen, datt dat gutt ass. (Je pense que c'est bien. / I think that that is good.)}
\end{itemize}

\subsection{Al vs. All}
\begin{itemize}
    \item \textbf{al} (adjectif / adjective): vieux / old
    \item \textbf{all} (indéfini / indefinite): tout, chaque / all, every
\end{itemize}

\subsection{Pronomen / Pronoms / Pronouns}
Pronouns change their form depending on their function in the sentence (case).
\textit{(Les pronoms changent de forme selon leur fonction dans la phrase (cas).)}

\begin{tabular}{llll}
\toprule
\textbf{Case} & \textbf{Nominative} & \textbf{Accusative} & \textbf{Dative} \\
\midrule
\textbf{2nd Pers. Sg. (you, informal)} & du & dech & dir \\
\textbf{2nd Pers. Pl.} & dir & iech & iech \\
\textbf{2nd Pers. Sg. (you, formal} & Dir & Iech & Iech \\
\bottomrule
\end{tabular}

\vspace{0.5em}
\textbf{Beispiller / Exemples:}
\begin{itemize}
    \item \textbf{Du} gesäis \textbf{dech} am Spigel. (Tu te vois dans le miroir. / You see yourself in the mirror.)
    \item Ech ginn \textbf{dir} e Buch. (Je te donne un livre. / I give you a book.)
    \item \textbf{Dir} gesitt \textbf{iech} am Spigel. (Vous vous voyez dans le miroir. / You see yourselves in the mirror.)
    \item Ech ginn \textbf{Iech} e Buch. (Je vous donne un livre. / I give you a book.)
\end{itemize}

\section{Wënsch a Virléiften / Souhaits et Préférences / Wishes and Preferences}

\begin{itemize}
    \item \textbf{Wat spills du gär?} \\
    \textit{(Qu'aimes-tu jouer ? / What do you like to play?)}
    
    \item \textbf{Wat wëlls du?} \\
    \textit{(Que veux-tu ? / What do you want?)}
    
    \item \textbf{Wat häss du gären?} \\
    \textit{(Que voudrais-tu ? / What would you like?)} \\
    \small{\textit{Polite form (Conditionnel) of "hunn" (to have) + "gären". / Forme polie (Conditionnel) de "hunn" (avoir) + "gären".}}
\end{itemize}