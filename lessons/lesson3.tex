\chapter{Leçon 3}

\section{Phrases Utiles (Useful Phrases)}
\begin{itemize}
    \item \textbf{Wat wëlls du haut wëssen?} \\
    \textit{(Qu'est-ce que tu veux savoir aujourd'hui?)} \\
    \textit{(What do you want to know today?)}
    
    \item \textbf{Ech brauch alles.} \\
    \textit{(J'ai besoin de tout.)} \\
    \textit{(I need everything.)}
    
    \item \textbf{Dat gefält mir net.} \\
    \textit{(Ça ne me plaît pas.)} \\
    \textit{(I don't like it.)}

    \item \textbf{Ech hunn dat net gär.} \\
    \textit{(Je n'aime pas ça.)} \\
    \textit{(I don't like that.)}
    
    \item \textbf{Ech schwätze gär mat dir.} \\
    \textit{(J'aime parler avec toi.)} \\
    \textit{(I like talking to you.)}
\end{itemize}

\section{Le Mot \textbf{Gléck} (Chance / Bonheur)}
Le mot \textbf{Gléck} est polyvalent en luxembourgeois. Il peut signifier la chance, le bonheur ou la fortune.

\begin{itemize}
    \item \textbf{Ech hu Gléck gehat.} \\
    \textit{(J'ai eu de la chance.)} \\
    \textit{(I was lucky.)}
    
    \item \textbf{Vill Gléck!} \\
    \textit{(Bonne chance!)} \\
    \textit{(Good luck!)}
    
    \item \textbf{Dat ass e grousst Gléck.} \\
    \textit{(C'est un grand bonheur.)} \\
    \textit{(That is a great happiness.)}
    
    \item \textbf{Zum Gléck si mir doheem.} \\
    \textit{(Heureusement, nous sommes à la maison.)} \\
    \textit{(Luckily, we are at home.)}
\end{itemize}

\section{Dire "Je t'aime" (Saying "I love you")}
Il existe plusieurs façons d'exprimer l'amour et l'affection en luxembourgeois, avec des nuances différentes.

\subsection*{A. Affection (L'amour amical / familial)}
\begin{itemize}
    \item \textbf{Ech hunn dech gär.} \\
    C'est la façon la plus courante et la plus polyvalente de dire "je t'aime bien" ou "je t'apprécie". Elle s'utilise pour les amis, la famille, et même au début d'une relation amoureuse. \\
    \textit{(This is the most common and versatile way to say "I like you" or "I'm fond of you". It is used for friends, family, and even at the beginning of a romantic relationship.)}
\end{itemize}

\subsection*{B. Amour Romantique (L'amour passionné)}
\begin{itemize}
    \item \textbf{Ech si frou mat dir.} \\
    Cette phrase se traduit littéralement par "Je suis content avec toi", mais elle exprime un sentiment plus profond de bonheur et de contentement dans une relation. C'est une déclaration d'amour sincère et établie. \\
    \textit{(Literally "I am happy with you", this phrase expresses a deeper feeling of happiness and contentment in a relationship. It is a sincere and established declaration of love.)}
\end{itemize}
