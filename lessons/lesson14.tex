\chapter{Leçon 14}

\section{Goen vs Ginn (Verben)}

\subsection*{Presens / Présent / Present Tense}

\subsection*{Goen (to go / aller)}
\begin{tabular}{ll c l l}
\toprule
\textbf{Pronom} & \textbf{Lëtzebuergesch} & \textbf{} & \textbf{Français} & \textbf{English} \\
\midrule
 ech & ginn & $\rightarrow$ & je vais & I go \\
 du & gees & $\rightarrow$ & tu vas & you go \\
 hien/si/hatt & geet & $\rightarrow$ & il/elle va & he/she/it goes \\
 mir & ginn & $\rightarrow$ & nous allons & we go \\
 dir & gitt & $\rightarrow$ & vous allez & you go \\
 si & ginn & $\rightarrow$ & ils/elles vont & they go \\
\bottomrule
\end{tabular}

\subsection*{Ginn (to give; to become / donner; devenir)}
\begin{tabular}{ll c l l}
\toprule
\textbf{Pronom} & \textbf{Lëtzebuergesch} & \textbf{} & \textbf{Français} & \textbf{English} \\
\midrule
 ech & ginn & $\rightarrow$ & je donne/deviens & I give/become \\
 du & gëss & $\rightarrow$ & tu donnes/deviens & you give/become \\
 hien/si/hatt & gëtt & $\rightarrow$ & il/elle donne/devient & he/she/it gives/becomes \\
 mir & ginn & $\rightarrow$ & nous donnons/devenons & we give/become \\
 dir & gitt & $\rightarrow$ & vous donnez/devenez & you give/become \\
 si & ginn & $\rightarrow$ & ils/elles donnent/deviennent & they give/become \\
\bottomrule
\end{tabular}

\section{Beispiller mat ``Ech ginn''}

\textit{``Ech ginn'' can mean ``I go'', ``I give'', or ``I become''. / ``Ech ginn'' peut signifier ``je vais'', ``je donne'' ou ``je deviens''.}

\begin{itemize}
    \item \textbf{Ech ginn Architekt.} \\
    (I become an architect. / Je deviens architecte.)
    \item \textbf{Ech ginn an de Park spazéieren.} \\
    (I go walking in the park. / Je vais me promener dans le parc.)
    \item \textbf{Ech ginn op d'Spillplaz spillen.} \\
    (I go playing on the playground. / Je vais jouer à l'aire de jeux.)
    \item \textbf{Ech ginn an de Supermarché akafen.} \\
    (I go shopping in the supermarket. / Je vais faire des courses au supermarché.)
    \item \textbf{Ech ginn dir Waasser.} \\
    (I give you water. / Je te donne de l'eau.)
\end{itemize}

\section{Nëtzlech Sätz (Phrases)}

\begin{itemize}
    \item \textbf{Gëff mir den Zocker.} \\
    (Give me the sugar. / Donne-moi le sucre.)
    \item \textbf{Gitt mir den Zocker, wannechgelift.} \\
    (Give me the sugar, please. (Formal/Plural) / Donnez-moi le sucre, s'il vous plaît.)
    \item \textbf{Kanns du mir den Zocker ginn?} \\
    (Can you give me the sugar? / Peux-tu me donner le sucre ?)
    \item \textbf{Kéint Dir mir den Zocker ginn?} \\
    (Could you give me the sugar? / Pourriez-vous me donner le sucre ?)
    \item \textbf{Kënnt Dir mir den Zocker ginn?} \\
    (Can you give me the sugar? (Formal) / Pouvez-vous me donner le sucre ?)
    \item \textbf{Et gëtt méi kal.} \\
    (It is getting colder. / Il commence à faire plus froid.)
    \item \textbf{Et gëtt vill Leit hei.} \\
    (There are many people here. / Il y a beaucoup de gens ici.)
    \item \textbf{Ëm wéi vill Auer geet d'Schoul un?} \\
    (At what time does school start? / À quelle heure commence l'école ?)
\end{itemize}

\section{Ufänken \& Undoen (Verben)}

\subsection*{Ufänken (to start / commencer)}
\begin{tabular}{ll c l l}
\toprule
\textbf{Pronom} & \textbf{Lëtzebuergesch} & \textbf{} & \textbf{Français} & \textbf{English} \\
\midrule
 ech & fänken un & $\rightarrow$ & je commence & I start \\
 du & fänks un & $\rightarrow$ & tu commences & you start \\
 hien/si/hatt & fänkt un & $\rightarrow$ & il/elle commence & he/she/it starts \\
 mir & fänken un & $\rightarrow$ & nous commençons & we start \\
 dir & fänkt un & $\rightarrow$ & vous commencez & you start \\
 si & fänken un & $\rightarrow$ & ils/elles commencent & they start \\
\bottomrule
\end{tabular}

\subsection*{Undoen / Undinn (to put on (clothes) / mettre)}
\begin{tabular}{ll c l l}
\toprule
\textbf{Pronom} & \textbf{Lëtzebuergesch} & \textbf{} & \textbf{Français} & \textbf{English} \\
\midrule
 ech & doen un / dinn un & $\rightarrow$ & je mets & I put on \\
 du & dees un & $\rightarrow$ & tu mets & you put on \\
 hien/si/hatt & deet un & $\rightarrow$ & il/elle met & he/she/it puts on \\
 mir & doen un / dinn un & $\rightarrow$ & nous mettons & we put on \\
 dir & dot un / ditt un & $\rightarrow$ & vous mettez & you put on \\
 si & doen un / dinn un & $\rightarrow$ & ils/elles mettent & they put on \\
\bottomrule
\end{tabular}

\subsection*{Beispill}

\begin{itemize}
    \item \textbf{Ech dinn/doen eng Jackett un.} \\
    (I put on a jacket. / Je mets une veste.)
\end{itemize}

\section{Opmaachen \& Zoumaachen}

\subsection*{Opmaachen (to open / ouvrir)}
\begin{tabular}{ll c l l}
\toprule
\textbf{Pronom} & \textbf{Lëtzebuergesch} & \textbf{} & \textbf{Français} & \textbf{English} \\
\midrule
 ech & maachen op & $\rightarrow$ & j'ouvre & I open \\
 du & méchs op & $\rightarrow$ & tu ouvres & you open \\
 hien/si/hatt & mécht op & $\rightarrow$ & il/elle ouvre & he/she/it opens \\
 mir & maachen op & $\rightarrow$ & nous ouvrons & we open \\
 dir & maacht op & $\rightarrow$ & vous ouvrez & you open \\
 si & maachen op & $\rightarrow$ & ils/elles ouvrent & they open \\
\bottomrule
\end{tabular}

\subsection*{Zoumaachen (to close / fermer)}
\begin{tabular}{ll c l l}
\toprule
\textbf{Pronom} & \textbf{Lëtzebuergesch} & \textbf{} & \textbf{Français} & \textbf{English} \\
\midrule
 ech & maachen zou & $\rightarrow$ & je ferme & I close \\
 du & méchs zou & $\rightarrow$ & tu fermes & you close \\
 hien/si/hatt & mécht zou & $\rightarrow$ & il/elle ferme & he/she/it closes \\
 mir & maachen zou & $\rightarrow$ & nous fermons & we close \\
 dir & maacht zou & $\rightarrow$ & vous fermez & you close \\
 si & maachen zou & $\rightarrow$ & ils/elles ferment & they close \\
\bottomrule
\end{tabular}

\subsection*{Beispiller}

\begin{itemize}
    \item \textbf{Maach d'Dier op, wannechgelift.} \\
    (Open the door, please. / Ouvre la porte, s'il te plaît.)
    \item \textbf{Ech maachen d'Fënster zou.} \\
    (I close the window. / Je ferme la fenêtre.)
\end{itemize}

\section{Sech erënneren (To remember)}

\subsection*{Sech erënneren (un) (to remember (sth/sb) / se souvenir (de))}
\begin{tabular}{ll c l l}
\toprule
\textbf{Pronom} & \textbf{Lëtzebuergesch} & \textbf{} & \textbf{Français} & \textbf{English} \\
\midrule
 ech & erënnere mech un & $\rightarrow$ & je me souviens de & I remember \\
 du & erënners dech un & $\rightarrow$ & tu te souviens de & you remember \\
 hien/si/hatt & erënnert sech un & $\rightarrow$ & il/elle se souvient de & he/she/it remembers \\
 mir & erënneren eis un & $\rightarrow$ & nous nous souvenons de & we remember \\
 dir & erënnert iech un & $\rightarrow$ & vous vous souvenez de & you remember \\
 si & erënneren sech un & $\rightarrow$ & ils/elles se souviennent de & they remember \\
\bottomrule
\end{tabular}

\subsection*{Beispiller}

\begin{itemize}
    \item \textbf{Ech erënnere mech un dech.} \\
    (I remember you. / Je me souviens de toi.)
    \item \textbf{Du erënners dech un eis.} \\
    (You remember us. / Tu te souviens de nous.)
    \item \textbf{Hien erënnert sech un iech.} \\
    (He remembers you (plural). / Il se souvient de vous.)
    \item \textbf{Mir erënneren eis un hien.} \\
    (We remember him. / Nous nous souvenons de lui.)
    \item \textbf{Dir erënnert iech un hatt.} \\
    (You remember her. / Vous vous souvenez d'elle.)
    \item \textbf{Si erënneren sech un d'Vakanz.} \\
    (They remember the vacation. / Ils se souviennent des vacances.)
\end{itemize}

\section{Illustréiert Wierder / Illustrated Words / Mots illustrés}

\begin{multicols}{2}

\begin{minipage}[t]{0.48\textwidth}
    \centering
    \textbf{d'Ram}
    \vspace{0.5em}
    \framebox{\parbox{0.9\linewidth}{\centering
            \vspace{0.5cm}
            \includegraphics[width=0.9\linewidth]{vokab_300dpi/ram.png} \\
            \small\textit{Crème / Cream}
            \vspace{0.5cm}
    }}
\end{minipage}
\hfill
\begin{minipage}[t]{0.48\textwidth}
    \centering
    \textbf{eng Ënn / eng Zwiwwel / eng Zwibbel}
    \vspace{0.5em}
    \framebox{\parbox{0.9\linewidth}{\centering
            \vspace{0.5cm}
            \includegraphics[width=0.9\linewidth]{vokab_300dpi/enn.png} \\
            \small\textit{Oignon / Onion} \\
            \textit{Pl: Ënnen / Zwiwwelen / Zwiwwelen}
            \vspace{0.5cm}
    }}
\end{minipage}

\vspace{0.5cm}

\begin{minipage}[t]{0.48\textwidth}
    \centering
    \textbf{eng Jackett}
    \vspace{0.5em}
    \framebox{\parbox{0.9\linewidth}{\centering
            \vspace{0.5cm}
            \includegraphics[width=0.9\linewidth]{vokab_300dpi/jackett.png} \\
            \small\textit{Veste / Jacket}
            \vspace{0.5cm}
    }}
\end{minipage}
\hfill
\begin{minipage}[t]{0.48\textwidth}
    \centering
    \textbf{den Zocker}
    \vspace{0.5em}
    \framebox{\parbox{0.9\linewidth}{\centering
            \vspace{0.5cm}
            \includegraphics[width=0.9\linewidth]{vokab_300dpi/zocker.png} \\
            \small\textit{Sucre / Sugar}
            \vspace{0.5cm}
    }}
\end{minipage}

\vspace{0.5cm}

\begin{minipage}[t]{0.48\textwidth}
    \centering
    \textbf{eng Dier}
    \vspace{0.5em}
    \framebox{\parbox{0.9\linewidth}{\centering
            \vspace{0.5cm}
            \includegraphics[width=0.9\linewidth]{vokab_300dpi/dier.png} \\
            \small\textit{Porte / Door}
            \vspace{0.5cm}
    }}
\end{minipage}

\end{multicols}

