\chapter{Leçon 5}

\section{Review of Verbs}

\subsection{Goen (to go)}
\begin{tabular}{ll c l l}
\toprule
\textbf{Pronom} & \textbf{Lëtzebuergesch} & \textbf{} & \textbf{Français} & \textbf{English} \\
\midrule
 ech & ginn & $\rightarrow$ & Je vais & I go \\
 du & gees & $\rightarrow$ & Tu vas & You go \\
 hien/hatt/si/et & geet & $\rightarrow$ & Il/Elle/On va & He/She/It goes \\
\midrule
 mir & ginn & $\rightarrow$ & Nous allons & We go \\
 dir & gitt & $\rightarrow$ & Vous allez & You go \\
 si & ginn & $\rightarrow$ & Ils/Elles vont & They go \\
\bottomrule
\end{tabular}

\subsection{Hunn (to have)}
\begin{tabular}{ll c l l}
\toprule
\textbf{Pronom} & \textbf{Lëtzebuergesch} & \textbf{} & \textbf{Français} & \textbf{English} \\
\midrule
 ech & hunn & $\rightarrow$ & J'ai & I have \\
 du & hues & $\rightarrow$ & Tu as & You have \\
 hien/hatt/si/et & huet & $\rightarrow$ & Il/Elle/On a & He/She/It has \\
\midrule
 mir & hunn & $\rightarrow$ & Nous avons & We have \\
 dir & hutt & $\rightarrow$ & Vous avez & You have \\
 si & hunn & $\rightarrow$ & Ils/Elles ont & They have \\
\bottomrule
\end{tabular}

\subsection{Sinn (to be)}
\begin{tabular}{ll c l l}
\toprule
\textbf{Pronom} & \textbf{Lëtzebuergesch} & \textbf{} & \textbf{Français} & \textbf{English} \\
\midrule
 ech & sinn & $\rightarrow$ & Je suis & I am \\
 du & bass & $\rightarrow$ & Tu es & You are \\
 hien/hatt/si/et & ass & $\rightarrow$ & Il/Elle/On est & He/She/It is \\
\midrule
 mir & sinn & $\rightarrow$ & Nous sommes & We are \\
 dir & sidd & $\rightarrow$ & Vous êtes & You are \\
 si & sinn & $\rightarrow$ & Ils/Elles sont & They are \\
\bottomrule
\end{tabular}

\section{Expressing Agreement and Giving Reasons}
En luxembourgeois, on peut exprimer son accord de plusieurs manières. "D'accord sinn" et "averstane sinn" signifient tous deux "être d'accord". "D'accord" est emprunté du français et est très courant. "Averstanen" est un mot plus germanique. Ils sont souvent interchangeables.
In Luxembourgish, you can express agreement in a few ways. "D'accord sinn" and "averstane sinn" both mean "to agree". "D'accord" is borrowed from French and is very common. "Averstanen" is a more Germanic word. They are often interchangeable.

The conjunction "well" (because) introduces a subordinate clause, which means the verb is sent to the end of the clause.

\begin{itemize}
    \item \textbf{Ech sinn d'accord, well du Recht hues.} \\
    \textit{(Je suis d'accord, parce que tu as raison. / I agree, because you are right.)}
    
    \item \textbf{Du bass d'accord, well ech Recht hunn.} \\
    \textit{(Tu es d'accord, parce que j'ai raison. / You agree, because I am right.)}

    \item \textbf{Hien ass d'accord, well hatt Recht huet.} \\
    \textit{(Il est d'accord, parce qu'elle a raison. / He agrees, because she is right.)}

    \item \textbf{Hatt ass d'accord, well si Recht hunn.} \\
    \textit{(Elle est d'accord, parce qu'ils ont raison. / She agrees, because they are right.)}

    \item \textbf{Si ass d'accord, well mir Recht hunn.} \\
    \textit{(Elle est d'accord, parce que nous avons raison. / She agrees, because we are right.)}

    \item \textbf{Mir sinn d'accord, well si Recht huet.} \\
    \textit{(Nous sommes d'accord, parce qu'elle a raison. / We agree, because she is right.)}

    \item \textbf{Dir sidd d'accord, well hien Recht huet.} \\
    \textit{(Vous êtes d'accord, parce qu'il a raison. / You agree, because he is right.)}
    
    \item \textbf{Si sinn d'accord, well dir Recht hutt.} \\
    \textit{(Ils sont d'accord, parce que vous avez raison. / They agree, because you are right.)}
\end{itemize}

\subsection{Asking "why" and giving a reason with "well"}
Remarquez la structure de la question et de la réponse. La question est une phrase standard avec le verbe en deuxième position. La réponse utilise "well" (parce que), donc le verbe de la proposition subordonnée va à la fin.
Notice the structure of the question and the answer. The question is a standard sentence with the verb in the second position. The answer uses "well" (because), so the verb of the subordinate clause goes to the end.
\begin{itemize}
    \item \textbf{Firwat gees du e Sonndeg de Moien um 8 Auer an d'Stad?} \\
    \textit{(Pourquoi vas-tu en ville dimanche matin à 8 heures? / Why are you going to the city on Sunday morning at 8 o'clock?)}
    
    \item \textbf{Ech ginn an d'Stad, well ech schwamme ginn.} \\
    \textit{(Je vais en ville parce que je vais nager. / I'm going to the city because I'm going swimming.)}
\end{itemize}
\textbf{Note on "well" clause word order:}
La règle grammaticale pour les clauses subordonnées introduites par "well" est que le verbe principal de cette clause va à la fin. Cependant, en luxembourgeois parlé, il est très courant d'utiliser un ordre des mots direct (sujet-verbe-objet) dans la clause "well", comme dans une phrase principale. Les deux formes sont acceptables, mais la forme avec le verbe à la fin est plus formelle et grammaticalement correcte.

The grammatical rule for subordinate clauses introduced by "well" is that the main verb of that clause goes to the end. However, in spoken Luxembourgish, it is very common to use a straight word ordering (subject-verb-object) in the "well" clause, similar to a main clause. Both forms are acceptable, but the verb-final form is more formal and grammatically correct.


\section{Expressing Disagreement}
Il y a plusieurs façons d'exprimer son désaccord, et il est important de comprendre les nuances.
There are a few ways to express disagreement, and it's important to understand the nuances.

\begin{itemize}
    \item \textbf{Ech sinn net d'accord.} \\
    \textit{(Je ne suis pas d'accord. / I don't agree.)}
    \item \textbf{Ech sinn net averstanen.} \\
    \textit{(Je ne suis pas d'accord. / I don't agree.)}
\end{itemize}
Il est également important de comprendre les phrases suivantes qui ont des significations différentes :
It is also important to understand the following phrases which have different meanings:
\begin{itemize}
    \item \textbf{Du hues net Recht.} \\
    \textit{(Tu n'as pas raison. / You are not right.)} \\
    C'est une manière directe et un peu abrupte de dire que quelqu'un se trompe sur les faits.
    This is a direct and somewhat blunt way of saying someone is factually incorrect.
    
    \item \textbf{Du hues kee Recht.} \\
    \textit{(Tu n'as pas le droit. / You have no right.)} \\
    C'est une déclaration plus forte, qui implique que la personne n'a aucune justification ou autorité pour dire ou faire quelque chose.
    This is a stronger statement, implying that the person has no justification or authority to say or do something.
\end{itemize}

\subsection{Related Words: Recht vs Rescht}
Il ne faut pas confondre les mots "Recht" et "Rescht". Bien que leur orthographe soit similaire, ils ont des significations complètement différentes.
Do not confuse the words "Recht" and "Rescht". Although their spelling is similar, they have completely different meanings.
\begin{itemize}
    \item \textbf{d'Recht} - (le droit, la loi / the right, the law) \\
    \textit{Dat ass däi Recht.} (C'est ton droit. / That is your right.)
    \item \textbf{de Rescht} - (le reste / the rest, the remainder) \\
    \textit{De Rescht ass fir dech.} (Le reste est pour toi. / The rest is for you.)
\end{itemize}

\section{Some Question Words}
\begin{itemize}
    \item \textbf{Wat?} - (Que/Quoi? / What?) \\
    \textit{Wat méchs du?} (Que fais-tu? / What are you doing?)
    \item \textbf{Wien?} - (Qui? / Who?) \\
    \textit{Wien ass dat?} (Qui est-ce? / Who is that?)
    \item \textbf{Wou?} - (Où? / Where?) \\
    \textit{Wou wunns du?} (Où habites-tu? / Where do you live?)
    \item \textbf{Wéini?} - (Quand? / When?) \\
    \textit{Wéini kënns du?} (Quand viens-tu? / When are you coming?)
    \item \textbf{Wéi?} - (Comment? / How?) \\
    \textit{Wéi geet et dir?} (Comment vas-tu? / How are you?)
    \item \textbf{Firwat?} - (Pourquoi? / Why?) \\
    \textit{Firwat laachs du?} (Pourquoi ris-tu? / Why are you laughing?)
    \item \textbf{Wéi vill?} - (Combien? / How much/many?) \\
    \textit{Wéi vill kascht dat?} (Combien ça coûte? / How much does that cost?)
\end{itemize}

\section{More Phrases and Structures}
\subsection{Talking about time and place}
\begin{itemize}
    \item \textbf{Wéini gees du an d'Vakanz?} \\
    \textit{(Quand pars-tu en vacances? / When are you going on vacation?)}
    
    \item \textbf{Ech ginn e Sonndeg um 8 Auer an d'Stad.} \\
    \textit{(Je vais en ville dimanche à 8 heures. / I'm going to the city on Sunday at 8 o'clock.)}
\end{itemize}
Dans ce contexte, \textbf{d'Stad} (la ville) se réfère presque toujours à la ville de Luxembourg.
In this context, \textbf{d'Stad} (the city) almost always refers to Luxembourg City.
Pour indiquer l'heure, \textbf{um} et \textbf{ëm} sont synonymes. Ils signifient tous deux "à" pour les heures exactes.
For telling time, \textbf{um} and \textbf{ëm} are synonyms. They both mean "at" for exact hours.

\section{Phrases and Dialogs}

\subsection{Going somewhere}
\begin{itemize}
    \item \textbf{Ech ginn akafen.} \\
    \textit{(Je vais faire des courses. / I'm going shopping.)}
\end{itemize}

\textbf{Dialog:}
\begin{itemize}
    \item Person A: \textbf{Gees du akafen?} \\
    \textit{(Vas-tu faire des courses? / Are you going shopping?)}
    \item Person B: \textbf{Nee, ech ginn net akafen.} \\
    \textit{(Non, je ne vais pas faire de courses. / No, I'm not going shopping.)}
\end{itemize}

\begin{itemize}
    \item \textbf{Ech gi schlofen.} \\
    \textit{(Je vais dormir. / I'm going to sleep.)}
    \item \textbf{Ech gi spadséieren.} \\
    \textit{(Je vais me promener. / I'm going for a walk.)}
    \item \textbf{Ech ginn iessen.} \\
    \textit{(Je vais manger. / I'm going to eat.)}
\end{itemize}

\subsection{Saying Good Night}
\begin{itemize}
    \item \textbf{Nuecht!} \\
    \textit{(Bonne nuit! / Good night!)}
\end{itemize}

\section{Vocabulaire (Vocabulary)}

\begin{minipage}[t]{0.48\textwidth}
    \centering
    \textbf{E Krees}
    \vspace{0.5em}
    \framebox{\parbox{0.9\linewidth}{\centering
            \vspace{0.5cm}
            \includegraphics[width=0.9\linewidth]{vokab_300dpi/Krees.png} \\
            \small\textit{Krees (Cercle / Circle)}
            \vspace{0.5cm}
    }}
\end{minipage}
\hfill
\begin{minipage}[t]{0.48\textwidth}
    \centering
    \textbf{Eng Gäns}
    \vspace{0.5em}
    \framebox{\parbox{0.9\linewidth}{\centering
            \vspace{0.5cm}
            \includegraphics[width=0.9\linewidth]{vokab_300dpi/Gäns.png} \\
            \small\textit{Gäns (Oie / Goose)}
            \vspace{0.5cm}
    }}
\end{minipage}
\hfill
\begin{minipage}[t]{0.48\textwidth}
    \centering
    \textbf{Eng Nuecht}
    \vspace{0.5em}
    \framebox{\parbox{0.9\linewidth}{\centering
            \vspace{0.5cm}
            \includegraphics[width=0.9\linewidth]{vokab_300dpi/Nuecht.png} \\
            \small\textit{Nuecht (Nuit / Night)}
            \vspace{0.5cm}
    }}
\end{minipage}
\hfill
\begin{minipage}[t]{0.48\textwidth}
    \centering
    \textbf{En Zuch}
    \vspace{0.5em}
    \framebox{\parbox{0.9\linewidth}{\centering
            \vspace{0.5cm}
            \includegraphics[width=0.9\linewidth]{vokab_300dpi/Zuch.png} \\
            \small\textit{Zuch (Train / Train)}
            \vspace{0.5cm}
    }}
\end{minipage}
\hfill
\begin{minipage}[t]{0.48\textwidth}
    \centering
    \textbf{Eng Geess}
    \vspace{0.5em}
    \framebox{\parbox{0.9\linewidth}{\centering
            \vspace{0.5cm}
            \includegraphics[width=0.9\linewidth]{vokab_300dpi/Geess.png} \\
            \small\textit{Geess (Chèvre / Goat)}
            \vspace{0.5cm}
    }}
\end{minipage}