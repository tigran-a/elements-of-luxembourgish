\chapter{Leçon 11}

\section{Besuch kréien / Receiving Guests / Recevoir des Visiteurs}

\begin{itemize}
    \item \textbf{Besuch kréien} \\
    \textit{(Recevoir de la visite / To receive guests)}
    \item \textbf{Ech kréien haut Besuch.} \\
    \textit{(J'ai de la visite aujourd'hui. / I am having guests today.)}
    \item \textbf{Komm eran!} (Informal) / \textbf{Kommt eran!} (Formal/Plural) \\
    \textit{(Entre ! / Entrez ! - Come in!)}
    \item \textbf{Maach der et bequem.} (Informal) / \textbf{Maacht Iech et bequem.} (Formal/Plural) \\
    \textit{(Fais comme chez toi. / Faites comme chez vous. - Make yourself comfortable.)}
    \item \textbf{Wëlls du eppes drénken?} \\
    \textit{(Veux-tu boire quelque chose ? / Do you want something to drink?)}
    \item \textbf{Setz dech!} (Informal) / \textbf{Setzt Iech!} (Formal/Plural) \\
    \textit{(Assieds-toi ! / Asseyez-vous ! - Sit down!)}
\end{itemize}

\section{Taart vs. Kuch / Pie vs. Cake / Tarte vs. Gâteau}

In Luxembourgish, there is a distinction between different types of baked sweets.
\textit{(En luxembourgeois, il y a une distinction entre différents types de pâtisseries.)}

\subsection{Taart (Tarte / Pie)}
Usually made with fruit on top of a crust.
\textit{(Généralement faite avec des fruits sur une pâte.)}

\begin{itemize}
    \item \textbf{Äppeltaart} (Tarte aux pommes / Apple pie)
    \item \textbf{Kiischtekuch} (Note: Cherry cake/pie often uses "Kuch" even if it looks like a tart, but generally fruit + crust = Taart)
    \item \textbf{Quetschentaart} (Tarte aux quetsches / Plum pie)
\end{itemize}

\subsection{Kuch (Gâteau / Cake)}
Usually a sponge cake, chocolate cake, or something mixed together.
\textit{(Généralement un gâteau éponge, au chocolat, ou un mélange.)}

\begin{itemize}
    \item \textbf{Schockelasskuch} (Gâteau au chocolat / Chocolate cake)
    \item \textbf{Marmorkuch} (Marbré / Marble cake)
    \item \textbf{Kéiskuch} (Gâteau au fromage / Cheesecake)
\end{itemize}

\section{Merci soen / Saying Thank You / Dire Merci}

\subsection{Merci soen / Dire merci / To say thank you}
\begin{itemize}
    \item \textbf{Merci.} (Merci. / Thank you.)
    \item \textbf{Villmools merci.} (Merci beaucoup. / Thank you very much.)
    \item \textbf{Merci villmools.} (Merci beaucoup. / Thank you very much.)
    \item \textbf{E grousse Merci.} (Un grand merci. / A big thank you.)
\end{itemize}

\subsection{Äntweren / Réponses / Responses}
\begin{itemize}
    \item \textbf{Gären.} (Volontiers/De rien. / You're welcome/My pleasure.)
    \item \textbf{Gär geschitt.} (Avec plaisir. / My pleasure.)  - \textit{ Used mostly if some efforts were spent on doing something (not simply giving something) /Utilisé principalement lorsque des efforts ont été consacrés à faire quelque chose (pas simplement à donner quelque chose).}
    \item \textbf{Wannechgelift.} (S'il vous plaît/De rien. / Please/You're welcome.) - \textit{ Used when giving something or responding to thanks. / Utilisé pour offrir quelque chose ou répondre à des remerciements.}
    \item \textbf{'t ass näicht.} (Ce n'est rien. / It's nothing.)
    \item \textbf{Kee Problem.} (Pas de problème. / No problem.)
\end{itemize}

\section{Fleeschzorten / Types of Meat / Types de Viande}

\begin{itemize}
    \item \textbf{Fleesch} (Viande / Meat)
    \item \textbf{Hingchen / Poulet} (Poulet / Chicken)
    \item \textbf{Schwéngefleesch} (Porc / Pork)
    \item \textbf{Rëndfleesch} (Bœuf / Beef)
    \item \textbf{Kalleffleesch} (Veau / Veal)
    \item \textbf{Ham} (Jambon / Ham)
    \item \textbf{Wurscht / Zoossiss} (Saucisse / Sausage)
\end{itemize}

\section{Akafe goen / Going Shopping / Faire les Courses}

\begin{itemize}
    \item \textbf{Akafe goen} (Faire les courses / To go shopping)
    \item \textbf{Ech gi spadséieren.} (Je vais me promener. / I am going for a walk.) - \textit{ Comparison structure}
    \item \textbf{Ech si gëschter akafe gaangen.} \\
    \textit{(Je suis allé(e) faire les courses hier. / I went shopping yesterday.)}
    \item \textbf{Du gees muer akafen.} \\
    \textit{(Tu vas faire les courses demain. / You are going shopping tomorrow.)}
    \item \textbf{Hie geet guer ni akafen, well hie keng Suen huet.} \\
    \textit{(Il ne va jamais faire les courses, car il n'a pas d'argent. / He never goes shopping because he has no money.)}
    \item \textbf{Mir mussen akafe goen.} \\
    \textit{(Nous devons aller faire les courses. / We must go shopping.)}
\end{itemize}

\section{Dictionary}
\begin{multicols}{2}

\begin{minipage}[t]{0.48\textwidth}
    \centering
    \textbf{eng Taart}
    \vspace{0.5em}
    \framebox{\parbox{0.9\linewidth}{\centering
            \vspace{0.5cm}
            \includegraphics[width=0.9\linewidth]{vokab_300dpi/taart.png} \\
            \small\textit{(tarte / pie)}
            \vspace{0.5cm}
    }}
\end{minipage}

\begin{minipage}[t]{0.48\textwidth}
    \centering
    \textbf{e Kuch}
    \vspace{0.5em}
    \framebox{\parbox{0.9\linewidth}{\centering
            \vspace{0.5cm}
            \includegraphics[width=0.9\linewidth]{vokab_300dpi/kuch.png} \\
            \small\textit{(gâteau / cake)}
            \vspace{0.5cm}
    }}
\end{minipage}

\begin{minipage}[t]{0.48\textwidth}
    \centering
    \textbf{Poulet}
    \vspace{0.5em}
    \framebox{\parbox{0.9\linewidth}{\centering
            \vspace{0.5cm}
            \includegraphics[width=0.9\linewidth]{vokab_300dpi/poulet.png} \\
            \small\textit{(poulet / chicken)}
            \vspace{0.5cm}
    }}
\end{minipage}

\begin{minipage}[t]{0.48\textwidth}
    \centering
    \textbf{Schwéngefleesch}
    \vspace{0.5em}
    \framebox{\parbox{0.9\linewidth}{\centering
            \vspace{0.5cm}
            \includegraphics[width=0.9\linewidth]{vokab_300dpi/schwengefleesch.png} \\
            \small\textit{(porc / pork)}
            \vspace{0.5cm}
    }}
\end{minipage}

\begin{minipage}[t]{0.48\textwidth}
    \centering
    \textbf{Rëndfleesch}
    \vspace{0.5em}
    \framebox{\parbox{0.9\linewidth}{\centering
            \vspace{0.5cm}
            \includegraphics[width=0.9\linewidth]{vokab_300dpi/rendfleesch.png} \\
            \small\textit{(bœuf / beef)}
            \vspace{0.5cm}
    }}
\end{minipage}

\begin{minipage}[t]{0.48\textwidth}
    \centering
    \textbf{Wurscht / Zoossiss}
    \vspace{0.5em}
    \framebox{\parbox{0.9\linewidth}{\centering
            \vspace{0.5cm}
            \includegraphics[width=0.9\linewidth]{vokab_300dpi/wurscht.png} \\
            \small\textit{(saucisse / sausage)}
            \vspace{0.5cm}
    }}
\end{minipage}


\begin{minipage}[t]{0.48\textwidth}
    \centering
    \textbf{en Hong}
    \vspace{0.5em}
    \framebox{\parbox{0.9\linewidth}{\centering
            \vspace{0.5cm}
            \includegraphics[width=0.9\linewidth]{vokab_300dpi/hong.png} \\
            \small\textit{(poulet (animal) / chicken (animal))}
            \vspace{0.5cm}
    }}
\end{minipage}

\begin{minipage}[t]{0.48\textwidth}
    \centering
    \textbf{en Hunn}
    \vspace{0.5em}
    \framebox{\parbox{0.9\linewidth}{\centering
            \vspace{0.5cm}
            \includegraphics[width=0.9\linewidth]{vokab_300dpi/hunn.png} \\
            \small\textit{(coq / rooster)}
            \vspace{0.5cm}
    }}
\end{minipage}

\begin{minipage}[t]{0.48\textwidth}
    \centering
    \textbf{eng Kou}
    \vspace{0.5em}
    \framebox{\parbox{0.9\linewidth}{\centering
            \vspace{0.5cm}
            \includegraphics[width=0.9\linewidth]{vokab_300dpi/kou.png} \\
            \small\textit{(vache / cow)}
            \vspace{0.5cm}
    }}
\end{minipage}

\begin{minipage}[t]{0.48\textwidth}
    \centering
    \textbf{e Stier}
    \vspace{0.5em}
    \framebox{\parbox{0.9\linewidth}{\centering
            \vspace{0.5cm}
            \includegraphics[width=0.9\linewidth]{vokab_300dpi/stier.png} \\
            \small\textit{(taureau / bull)}
            \vspace{0.5cm}
    }}
\end{minipage}

\begin{minipage}[t]{0.48\textwidth}
    \centering
    \textbf{e Schwäin}
    \vspace{0.5em}
    \framebox{\parbox{0.9\linewidth}{\centering
            \vspace{0.5cm}
            \includegraphics[width=0.9\linewidth]{vokab_300dpi/schwaein.png} \\
            \small\textit{(cochon / pig)}
            \vspace{0.5cm}
    }}
\end{minipage}

\begin{minipage}[t]{0.48\textwidth}
    \centering
    \textbf{e Béier}
    \vspace{0.5em}
    \framebox{\parbox{0.9\linewidth}{\centering
            \vspace{0.5cm}
            \includegraphics[width=0.9\linewidth]{vokab_300dpi/boar_beier.png} \\
            \small\textit{(verrat / boar (male pig))}
            \vspace{0.5cm}
    }}
\end{minipage}

\end{multicols}